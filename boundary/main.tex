% !Mode:: "TeX:UTF-8"
\documentclass[12pt,a4paper]{article}
\input{../en_preamble.tex}
\input{../xecjk_preamble.tex}

\title{边界处理}
%\author{ }
\date{\chntoday}

\begin{document}
\maketitle
没有经过边界处理的刚度矩阵一般是奇异的,因此要对刚度矩阵进行边界处理,这里讨论狄利克雷边界。

\section{一维边界处理}

\begin{figure}[H]
\centering
\includegraphics[scale=0.5]{./figures/1.png}
\caption{}
\end{figure}

假设没有经过边界处理的刚度矩阵$A$:
$$
\begin{bmatrix}
a_{00} & a_{01} & a_{02} & \cdots & a_{0,n-1} & a_{0n} \\
a_{10} & a_{11} & a_{12} & \cdots & a_{1,n-1} & a_{1n} \\
a_{20} & a_{21} & a_{22} & \cdots & a_{2,n-1} & a_{2n} \\
\vdots & \vdots & \vdots &  & \vdots  & \vdots \\
a_{n-2,0} & a_{n-2,1} & a_{n-2,2} & \cdots & a_{n-2,n-1} & a_{n-2,n} \\
a_{n-1,0} & a_{n-1,1} & a_{n-1,2} & \cdots & a_{n-1,n-1} & a_{n-1,n} \\
a_{n0} & a_{n1} & a_{n2} & \cdots & a_{n,n-1} & a_{nn} 
\end{bmatrix}
$$

即$$
\begin{bmatrix}
a_{00} & a_{01} & a_{02} & \cdots & a_{0,n-1} & a_{0n} \\
a_{10} & a_{11} & a_{12} & \cdots & a_{1,n-1} & a_{1n} \\
a_{20} & a_{21} & a_{22} & \cdots & a_{2,n-1} & a_{2n} \\
\vdots & \vdots & \vdots &  & \vdots  & \vdots \\
a_{n-2,0} & a_{n-2,1} & a_{n-2,2} & \cdots & a_{n-2,n-1} & a_{n-2,n} \\
a_{n-1,0} & a_{n-1,1} & a_{n-1,2} & \cdots & a_{n-1,n-1} & a_{n-1,n} \\
a_{n0} & a_{n1} & a_{n2} & \cdots & a_{n,n-1} & a_{nn} 
\end{bmatrix}
\begin{bmatrix}
u_{0} \\
u_{1} \\
u_{2} \\
\vdots \\
u_{n-2} \\
u_{n-1} \\
u_{n} \\
\end{bmatrix}=
\begin{bmatrix}
b_{0} \\
b_{1} \\
b_{2} \\
\vdots \\
b_{n-2} \\
b_{n-1} \\
b_{n} \\
\end{bmatrix}$$

其中,$B=\begin{bmatrix}
b_{0} \\
b_{1} \\
b_{2} \\
\vdots \\
b_{n-2} \\
b_{n-1} \\
b_{n} \\
\end{bmatrix}$是没有经过边界处理的载荷向量,$u_0$ 和 $u_n$是$u$在边界上的已知值。

因此有

$$
\begin{cases}
a_{00}u_0 + a_{01}u_1 + a_{02}u_2 +\cdots + a_{0,n-1}u_{n-1} + a_{0n}u_n=b_0 \\
a_{10}u_0 + a_{11}u_1 + a_{12}u_2 +\cdots + a_{1,n-1}u_{n-1} + a_{1n}u_n=b_1 \\
a_{20}u_0 + a_{21}u_1 + a_{22}u_2 +\cdots + a_{2,n-1}u_{n-1} + a_{2n}u_n=b_2 \\
\cdots \\
a_{n-2,0}u_0 + a_{n-2,1}u_1 + a_{n-2,2}u_2 +\cdots + a_{n-2,n-1}u_{n-1} + a_{n-2,n}u_n=b_{n-2} \\
a_{n-1,0}u_0 + a_{n-1,1}u_1 + a_{n-1,2}u_2 +\cdots + a_{n-1,,n-1}u_{n-1} + a_{n-1,n}u_n=b_{n-1} \\
a_{n0}u_0 + a_{n1}u_1 + a_{n2}u_2 +\cdots + a_{n,n-1}u_{n-1} + a_{nn}u_n=b_n
\end{cases}
$$

现在开始对 $A$ 进行边界处理,首先找到边界点所在的位置,然后把 $A$ 相应的元素变成对角线上为1,对角线元素其他的行和列变成0,其余元素不变。这里0和 $n$ 是边界点,因此
$$
A=\begin{bmatrix}
1 & 0 & 0 & \cdots & 0 & 0 \\
0 & a_{11} & a_{12} & \cdots & a_{1,n-1} & 0 \\
0 & a_{21} & a_{22} & \cdots & a_{2,n-1} & 0 \\
\vdots & \vdots & \vdots &  & \vdots  & \vdots \\
0 & a_{n-2,1} & a_{n-2,2} & \cdots & a_{n-2,n-1} & 0 \\
0 & a_{n-1,1} & a_{n-1,2} & \cdots & a_{n-1,n-1} & 0 \\
0 & 0 & 0 & \cdots & 0 & 1 
\end{bmatrix}
$$

$A$ 发生了改变,因此 $B$ 也应当发生改变。

把上面的公式重写成下面的形式

$$
\begin{cases}
u_0 + 0 + 0 +\cdots + 0 + 0=u_0 \\
a_{11}u_1 + a_{12}u_2 +\cdots + a_{1,n-1}u_{n-1}=b_1-a_{10}u_0-a_{1n}u_n \\
 a_{21}u_1 + a_{22}u_2 +\cdots + a_{2,n-1}u_{n-1}=b_2-a_{20}u_0-a_{2n}u_n\\
\cdots \\
a_{n-2,1}u_1 + a_{n-2,2}u_2 +\cdots + a_{n-2,n-1}u_{n-1}=b_{n-2}-a_{n-2,0}u_0-a_{n-2,n}u_n \\
a_{n-1,1}u_1 + a_{n-1,2}u_2 +\cdots + a_{n-1,,n-1}u_{n-1}=b_{n-1}-a_{n-1,0}u_0-a_{n-1,n}u_n \\
0 + 0 +\cdots + 0 + u_n=u_n
\end{cases}
$$

即
$$
\begin{bmatrix}
1 & 0 & 0 & \cdots & 0 & 0 \\
0 & a_{11} & a_{12} & \cdots & a_{1,n-1} & 0 \\
0 & a_{21} & a_{22} & \cdots & a_{2,n-1} & 0 \\
\vdots & \vdots & \vdots &  & \vdots  & \vdots \\
0 & a_{n-1,1} & a_{n-1,2} & \cdots & a_{n-1,n-1} & 0 \\
0 & 0 & 0 & \cdots & 0 & 1
\end{bmatrix}
\begin{bmatrix}
u_{0} \\
u_{1} \\
u_{2} \\
\vdots \\
u_{n-1} \\
u_{n} \\
\end{bmatrix}=
\begin{bmatrix}
u_{0} \\
b_1-a_{10}u_0-a_{1n}u_n \\
b_2-a_{20}u_0-a_{2n}u_n \\
\vdots \\
b_{n-2}-a_{n-2,0}u_0-a_{n-2,n}u_n  \\
b_{n-1}-a_{n-1,0}u_0-a_{n-1,n}u_n \\
u_{n} \\
\end{bmatrix}
$$

其中
$$B=\begin{bmatrix}
u_{0} \\
b_1-a_{10}u_0-a_{1n}u_n \\
b_2-a_{20}u_0-a_{2n}u_n \\
\vdots \\
b_{n-2}-a_{n-2,0}u_0-a_{n-2,n}u_n  \\
b_{n-1}-a_{n-1,0}u_0-a_{n-1,n}u_n \\
u_{n} \\
\end{bmatrix}
$$
是经过边界处理后的载荷向量

\section{二维边界处理}

这里为方便讨论,假设是下面这种情况

\begin{figure}[H]
\centering
\includegraphics[scale=0.5]{./figures/2.png}
\caption{}
\end{figure}

假设没有经过边界处理的刚度矩阵$A$:
$$
\begin{bmatrix}
a_{00} & a_{01} & a_{02} & a_{03} & a_{04} & a_{05} & a_{06} & a_{07} & a_{08}\\
a_{10} & a_{11} & a_{12} & a_{13} & a_{14} & a_{15} & a_{16} & a_{17} & a_{18}\\
a_{20} & a_{21} & a_{22} & a_{23} & a_{24} & a_{25} & a_{26} & a_{27} & a_{28}\\
a_{30} & a_{31} & a_{32} & a_{33} & a_{34} & a_{35} & a_{36} & a_{37} & a_{38}\\
a_{40} & a_{41} & a_{42} & a_{43} & a_{44} & a_{45} & a_{46} & a_{47} & a_{48}\\
a_{50} & a_{51} & a_{52} & a_{53} & a_{54} & a_{55} & a_{56} & a_{57} & a_{58}\\
a_{60} & a_{61} & a_{62} & a_{63} & a_{64} & a_{65} & a_{66} & a_{67} & a_{68}\\
a_{70} & a_{71} & a_{72} & a_{73} & a_{74} & a_{75} & a_{76} & a_{77} & a_{78}\\
a_{80} & a_{81} & a_{82} & a_{83} & a_{84} & a_{85} & a_{86} & a_{87} & a_{88}\\
\end{bmatrix}
$$

即$$
\begin{bmatrix}
a_{00} & a_{01} & a_{02} & a_{03} & a_{04} & a_{05} & a_{06} & a_{07} & a_{08}\\
a_{10} & a_{11} & a_{12} & a_{13} & a_{14} & a_{15} & a_{16} & a_{17} & a_{18}\\
a_{20} & a_{21} & a_{22} & a_{23} & a_{24} & a_{25} & a_{26} & a_{27} & a_{28}\\
a_{30} & a_{31} & a_{32} & a_{33} & a_{34} & a_{35} & a_{36} & a_{37} & a_{38}\\
a_{40} & a_{41} & a_{42} & a_{43} & a_{44} & a_{45} & a_{46} & a_{47} & a_{48}\\
a_{50} & a_{51} & a_{52} & a_{53} & a_{54} & a_{55} & a_{56} & a_{57} & a_{58}\\
a_{60} & a_{61} & a_{62} & a_{63} & a_{64} & a_{65} & a_{66} & a_{67} & a_{68}\\
a_{70} & a_{71} & a_{72} & a_{73} & a_{74} & a_{75} & a_{76} & a_{77} & a_{78}\\
a_{80} & a_{81} & a_{82} & a_{83} & a_{84} & a_{85} & a_{86} & a_{87} & a_{88}\\
\end{bmatrix}
\begin{bmatrix}
u_{0} \\
u_{1} \\
u_{2} \\
u_{3} \\
u_{4} \\
u_{5} \\
u_{6} \\
u_{7} \\
u_{8} \\
\end{bmatrix}=
\begin{bmatrix}
b_{0} \\
b_{1} \\
b_{2} \\
b_{3} \\
b_{4} \\
b_{5} \\
b_{6} \\
b_{7} \\
b_{8} \\
\end{bmatrix}$$

其中,$B=\begin{bmatrix}
b_{0} \\
b_{1} \\
b_{2} \\
b_{3} \\
b_{4} \\
b_{5} \\
b_{6} \\
b_{7} \\
b_{8} \\
\end{bmatrix}$是没有经过边界处理的载荷向量,$u_0,u_1,u_2,u_3,u_5,u_6,u_7,u_8$是$u$在边界上的已知值。因此有

$$
\begin{cases}
a_{00}u_0 + a_{01}u_1 + a_{02}u_2 + a_{03}u_3 + a_{04}u_{4} + a_{05}u_5 + a_{06}u_6 + a_{07}u_7 + a_{08}u_8=b_0 \\
a_{10}u_0 + a_{11}u_1 + a_{12}u_2 + a_{13}u_3 + a_{14}u_{4} + a_{15}u_5 + a_{16}u_6 + a_{17}u_7 + a_{18}u_8=b_1 \\
a_{20}u_0 + a_{21}u_1 + a_{22}u_2 + a_{23}u_3 + a_{24}u_{4} + a_{25}u_5 + a_{26}u_6 + a_{27}u_7 + a_{28}u_8=b_2 \\
a_{30}u_0 + a_{31}u_1 + a_{32}u_2 + a_{33}u_3 + a_{34}u_{4} + a_{35}u_5 + a_{36}u_6 + a_{37}u_7 + a_{38}u_8=b_3 \\
a_{40}u_0 + a_{41}u_1 + a_{42}u_2 + a_{43}u_3 + a_{44}u_{4} + a_{45}u_5 + a_{46}u_6 + a_{47}u_7 + a_{48}u_8=b_4 \\
a_{50}u_0 + a_{51}u_1 + a_{52}u_2 + a_{53}u_3 + a_{54}u_{4} + a_{55}u_5 + a_{56}u_6 + a_{57}u_7 + a_{58}u_8=b_5 \\
a_{60}u_0 + a_{61}u_1 + a_{62}u_2 + a_{63}u_3 + a_{64}u_{4} + a_{65}u_5 + a_{66}u_6 + a_{67}u_7 + a_{68}u_8=b_6 \\
a_{70}u_0 + a_{71}u_1 + a_{72}u_2 + a_{73}u_3 + a_{74}u_{4} + a_{75}u_5 + a_{76}u_6 + a_{77}u_7 + a_{78}u_8=b_7 \\
a_{80}u_0 + a_{81}u_1 + a_{82}u_2 + a_{83}u_3 + a_{84}u_{4} + a_{85}u_5 + a_{86}u_6 + a_{87}u_7 + a_{88}u_8=8_0 \\
\end{cases}
$$

现在开始对 $A$ 进行边界处理,和一维边界处理一样,首先找到边界点所在的位置,然后把 $A$ 相应的元素变成对角线上为1,对角线元素其他的行和列变成0,其余元素不变。这里$0,1,2,3,5,6,7,8$是边界点,因此

$$
A=\begin{bmatrix}
1 & 0 & 0 & 0 & 0 & 0 & 0 & 0 & 0 \\
0 & 1 & 0 & 0 & 0 & 0 & 0 & 0 & 0 \\
0 & 0 & 1 & 0 & 0 & 0 & 0 & 0 & 0 \\
0 & 0 & 0 & 1 & 0 & 0 & 0 & 0 & 0 \\
0 & 0 & 0 & 0 & a_{44} & 0 & 0 & 0 & 0\\
0 & 0 & 0 & 0 & 0 & 1 & 0 & 0 & 0\\
0 & 0 & 0 & 0 & 0 & 0 & 1 & 0 & 0 \\
0 & 0 & 0 & 0 & 0 & 0 & 0 & 1 & 0 \\
0 & 0 & 0 & 0 & 0 & 0 & 0 & 0 & 1 \\
\end{bmatrix}
$$

$A$ 发生了改变,因此 $B$ 也应当发生改变。

把上面的公式重写成下面的形式

$$
\begin{cases}
u_0 + 0 + 0 + 0 + 0 + 0 + 0 + 0 + 0 = u_0 \\
0 + u_1 + 0 + 0 + 0 + 0 + 0 + 0 + 0 = u_1 \\
0 +  0 + u_2 +0 + 0 + 0 + 0 + 0 + 0 = u_2 \\
0 +  0 + 0 + u_3 +0 + 0 + 0 + 0 + 0 = u_3 \\
a_{44}u_{4} = b_4 -a_{40}u_0 - a_{41}u_1- a_{42}u_2-a_{43}u_3 -a_{45}u_5- a_{46}u_6- a_{47}u_7- a_{48}u_8 \\
0 +  0 + 0 + 0 + u_5 +0 + 0 + 0 + 0 = u_5 \\
0 +  0 + 0 + 0 + 0 +u_6 +0 + 0 + 0 = u_6 \\
0 +  0 + 0 + 0 + 0 + 0 + u_7 +0 + 0 = u_7 \\
0 +  0 + 0 + 0 + 0 + 0 + 0 + u_8 + 0 = u_8 \\
\end{cases}
$$

即
$$
\begin{bmatrix}
1 & 0 & 0 & 0 & 0 & 0 & 0 & 0 & 0 \\
0 & 1 & 0 & 0 & 0 & 0 & 0 & 0 & 0 \\
0 & 0 & 1 & 0 & 0 & 0 & 0 & 0 & 0 \\
0 & 0 & 0 & 1 & 0 & 0 & 0 & 0 & 0 \\
0 & 0 & 0 & 0 & a_{44} & 0 & 0 & 0 & 0\\
0 & 0 & 0 & 0 & 0 & 1 & 0 & 0 & 0\\
0 & 0 & 0 & 0 & 0 & 0 & 1 & 0 & 0 \\
0 & 0 & 0 & 0 & 0 & 0 & 0 & 1 & 0 \\
0 & 0 & 0 & 0 & 0 & 0 & 0 & 0 & 1 \\
\end{bmatrix}
\begin{bmatrix}
u_{0} \\
u_{1} \\
u_{2} \\
u_{3} \\
u_{4} \\
u_{5} \\
u_{6} \\
u_{7} \\
u_{8} \\
\end{bmatrix}=\\
\begin{bmatrix}
u_{0} \\
u_{1} \\
u_{2} \\
u_{3} \\
b_4-a_{40}u_0-a_{41}u_1-a_{42}u_2-a_{43}u_3-a_{45}u_5-a_{46}u_6-a_{47}u_7-a_{48}u_8
u_{5} \\
u_{6} \\
u_{7} \\
u_{8} \\
\end{bmatrix}
$$

其中
$$
B=\begin{bmatrix}
u_{0} \\
u_{1} \\
u_{2} \\
u_{3} \\
b_4-a_{40}u_0-a_{41}u_1-a_{42}u_2-a_{43}u_3-a_{45}u_5-a_{46}u_6-a_{47}u_7-a_{48}u_8
u_{5} \\
u_{6} \\
u_{7} \\
u_{8} \\
\end{bmatrix}
$$
是经过边界处理后的载荷向量



































%\input{test.tex}

%\cite{tam19912d}
%\bibliography{../ref}
\end{document}
