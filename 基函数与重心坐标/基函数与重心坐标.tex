% !Mode:: "TeX:UTF-8"
\documentclass{article}
\input{en_preamble.tex}
\input{xecjk_preamble.tex}
\setCJKmainfont{STKaiti} % 如果请替换为本地系统有的字体
%中文断行
\XeTeXlinebreaklocale "zh"
\XeTeXlinebreakskip = 0pt plus 1pt minus 0.1pt
\begin{document}
\title{变分整理}
%\date{\today}
\maketitle
\tableofcontents
%\newpage
\section{变分定义}
函数$y(x)$的变分:在同一定义域上的两个函数$y(x),m(x)$,若彼此任意接近,那么$m(x)$与$y(x)$之差,记$\delta y(x)=m(x)-y(x)$称$y(x)$的变分。

变分$\delta y$反映了整个函数的变化,而函数增量$ \Delta y $反应的是同一个函数由于自变量的取值不同引起的变化。

泛函的变分:对任意一个泛函$J[y]$,函数$y$引起的泛函增量为$\Delta J=J[y+\delta y]-J[y]=L[y,\delta y]+\dfrac{1}{2!}Q[y,\delta y]+o(||\delta y||^2)$

其中$L(y,\delta y)$是关于$\delta y$的线性泛函,$Q[y,\delta y]$是关于$\delta y$的两次线性泛函。

则泛函的一阶变分$\delta J=L[y,\delta y]$,二阶变分$\delta^2J=Q[y,\delta y]$

变分的另一种求法,任给一个齐次函数$\eta(x)$,边界值为零,对任意小的一个实数$\varepsilon$,$y^{*}(x)=y(x)+\varepsilon\eta(x)$当$\varepsilon=0$

$\delta J=\varepsilon\frac{dJ[y+\varepsilon\eta]}{d\varepsilon}$

$\delta^2 J=\frac{1}{2!}\varepsilon^2\frac{d^2J[y+\varepsilon\eta]}{d\varepsilon^2}$

\section{泛函例子}
例:已知泛函$J[y]=\int_{a}^{b}F(x,y,y^{'})\,\mathrm{d}x$

$ \delta J[y]=\int_{a}^{b}[\frac{\partial F}{\partial y}\delta y+\frac{\partial F}{\partial y^{'}}\delta y^{'}]\,\mathrm{d}x $

$=\frac{\partial F}{\partial y^{'}}(\delta y)+\int_{a}^{b}[\frac{\partial F}{\partial y}+\dfrac{d}{dx}(\frac{\partial F}{\partial y^{'}})]\delta y\,\mathrm{d}x $

例:最短线问题,设经过$A,B$两点距离最短曲线方程为$y^{*}(x)$,另有任意连续可导函数$\eta=\eta(x)$,且$\eta(x)$满足$\eta(x_{0})=\eta(x_{1})=0$,则$y=y^{*}+\alpha\eta(x)$仍是过$A,B$的连续曲线,其对应长度为$L(a)=\int_{x_{0}}^{x_{1}}\sqrt{1+(y^{*})^{'}+\alpha\eta^{'})^2}\,\mathrm{d}x$

当$a=0$,$y=y^{*}(x)$时$L(a)$取极小值,即$\frac{dL(a)}{da}=0$在$a=0$的地方

$\frac{dL(a)}{da}=\int_{x_{0}}^{x_{1}}\frac{((y^{*})^{'}+\alpha\eta^{'})\eta^{'}}{\sqrt{1+((y^{*})^{'})^2}}\,\mathrm{d}x$

$=\int_{x_{0}}^{x_{1}}\frac{(y^{*})^{'}\eta^{'}}{\sqrt{1+(y^{*})^2}}\,\mathrm{d}x$ 

$=\frac{(y^{*})^{'}\eta}{\sqrt{1+(y^{*})^2}}-\int_{x_{0}}^{x_{1}}(\frac{(y^{*})^{'}}{\sqrt{1+((y^{*})^{'})^2}})^{'}\eta\,\mathrm{d}x$

$=-\int_{x_{0}}^{x_{1}}(\frac{(y^{*})^{''}}{\sqrt{1+(y^{*})^2}}-\frac{(y^{*})^{'} (y^{*})^{'}(y^{*}){''}}{\sqrt{1+(y^{*})^3}}))\eta\,\mathrm{d}x$

$=-\int_{x_{0}}^{x_{1}}\frac{y^{''}}{\sqrt{1+((y^{*})^{2})^3}}\eta\,\mathrm{d}x$

由$\eta(x)$的任意性,由变分引理可得$\frac{y^{''}}{\sqrt{1+((y^{*})^{2}})^{3}}=0$
%\begin{equation}
%-\nabla \cdot (\beta\nabla u) = f(x,y),\,\ (x,y)\in \Omega
%\end{equation}
%Dilichlet 边界条件

%\begin{equation}
%u(x,y) = g(x,y),\,\ (x,y)\in \partial \Omega
%\end{equation}
%\subsection{符号}
%\begin{tabular}{ |l|l| }   
%\hline   
%\multicolumn{2}{|c|}{符号说明} \\   
%\hline
%符号 & 含义\\
%\hline
%$\Omega$ & 二维长方形区域 \\
%\hline
%$nx$ & $x$ 方向剖分的段数 \\
%\hline
%$ny$ & $y$ 方向剖分的段数 \\
%\hline
%$hx$ &  $x$ 方向每段的长度\\
%\hline
%$hy$ &  $y$ 方向每段的长度 \\
%\hline
%$\mu$ & $the \,\ viscosity \,\ coefficient$ \\
%\hline
%$k$ & $the \,\ permeability \,\ tensor$ \\
%\hline 
%$NC$ & 代表 $cell$ 的个数 \\
%\hline
%$NE$ & 代表总的 $edge$ 的个数 \\
%\hline
%\end{tabular}

%\section{模型}
%\begin{equation*}
%\begin{cases}
%\begin{aligned}
%\frac{\mu}{k}\mathbf{u} + \nabla p & = 0 \quad in \,\ \Omega = (0,1)\times (0,1) \\
%\nabla \cdot \mathbf{u} & = f \quad in \,\ \Omega \\
%\mathbf {u} & = 0 \quad on \,\ \partial \Omega
%\end{aligned}
%\end{cases}
%\end{equation*}

%且有 \\
%\begin{equation*}
%\int_{\Omega}f dxdy = 0
%\end{equation*}

%记 $u$ 为 $\mathbf{u}$ 在 $x$ 方向的分量,$v$ 为 $\mathbf{u}$ 在 $y$ 方向的分量,则有 \\

%\begin{equation*}
%\begin{cases}
%\begin{aligned}
%\frac{\mu}{k}\cdot u + \partial_x p & = 0 \quad (1) \\
%\frac{\mu}{k}\cdot v + \partial_y p & = 0 \quad (2) \\
%\partial_x u + \partial_y v & = f \quad (3)
%\end{aligned}
%\end{cases}
%\end{equation*}

%\section{离散后组装矩阵}
%利用一阶向前差分把方程变成差分方程,现在从 $edge$ 和 $cell$ 的角度考虑模型。 \\

%对于 $(1)$, 从内部纵向 $edge$ 的角度考虑:
%我们需要找到内部纵向 $edge$ 所对应的左手边的 $cell$ 和右手边的 $cell$. 左右两边的$cell$ 所对应的 $p$ 分别记为 $p_{l}$、$p_{r}$.$u$ 为 $edge$ 的中点,记为 $u_m$。按照 $mesh$ 里的编号规则排序。\\

%则每条内部边上所对应的差分方程为:

%\begin{equation*}
%\frac{\mu}{k} \cdot u_m + \frac{p_r - p_l}{hx} = 0
%\end{equation*}

%对于 $(2)$,从内部横向 $edge$ 的角度考虑:
%我们需要找到内部横向 $edge$ 所对应的左手边的 $cell$ 和右手边的 $cell$. $cell$ 所对应的 $p$ 与 $(1)$ 中的相同。$v$ 为 $edge$ 的中点,记为 $v_m$。\\

%则每条内部边上所对应的差分方程为:\\

%\begin{equation*}
%\frac{\mu}{k} \cdot v_m + \frac{p_l - p_r}{hy} = 0
%\end{equation*}

%对于 $(3)$, 从 $cell$ 的角度考虑:
%由于单元是四边形单元,我们记单元所对应边的局部编号为[0,1,2,3](StructureQuadMesh.py 里的网格),第 $i$ 个单元所对应的边记为 $e_{i,0},e_{i,1},e_{i,2},e_{i,3}$。\\

%则 $(3)$ 式第 $i$ 个单元所对应的差分方程为:\\

%\begin{equation*}
%\frac{u_{e_{i,1}} - u_{e_{i,3}}}{hx} + \frac{v_{e_{i,2}} - v_{e_{i,0}}}{hy} = f_i
%\end{equation*}

%我们需要生成一个 $(NE+NC)\times(NE+NC)$的系数矩阵,把它看成分块矩阵
%\begin{equation*}
%\begin{pmatrix}
% A_{1,1} & A_{1,2} \\
%A_{2,1} & A_{2,2}
%\end{pmatrix}
%\end{equation*}

%其中 \\

%\begin{equation*}
%\begin{aligned}
%A_{1,1} : NE\times NE \\
%A_{1,2} : NE\times NC \\
%A_{2,1} : NC\times NE \\
%A_{2,2} : NC\times NC
%\end{aligned}
%\end{equation*}

%$A_{1,1}$ 对应的是 $(1),(2)$ 两式的第一项,即含有 $u,v$ 的项,$A_12$ 对应的是 $(1),(2)$ 两式的第二项。

%\newpage
%\nocite{*}
%\bibliography{ref}
\end{document}

