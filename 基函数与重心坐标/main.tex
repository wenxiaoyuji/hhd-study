% !Mode:: "TeX:UTF-8"
\documentclass{article}
\input{en_preamble.tex}
\input{xecjk_preamble.tex}
\setCJKmainfont{STKaiti} % 如果请替换为本地系统有的字体
%中文断行
\XeTeXlinebreaklocale "zh"
\XeTeXlinebreakskip = 0pt plus 1pt minus 0.1pt
\begin{document}
\title{重心坐标与基函数}
%\date{\today}
\maketitle
%\tableofcontents
%\newpage
\section{重心坐标}
重心坐标是构造基函数的基础,所以首先引入重心坐标。

数学中,重心坐标是由单形(如三角形或四面体等)顶点定义的坐标。经典的重心坐标是一种定义在“多边形”上的坐标,而不局限于具体的坐标系。“多边形”内的点由“多边形”各顶点线性表出,组合系数便是重心坐标。下面定义重心坐标,以平面上的多边形为例。

令 $P$ 是一个平面上的 $n$ 边形,其顶点记为 $\lbrace X_i\rbrace_{i=0,\cdots,n}$,~$n\ge 2$,且以逆时针顺序标记。对任意的 $X\in P$,定义函数 $\lambda _i(x):P\longrightarrow R,~i=0,\cdots,n$.若

$$
X=\sum_{i=0}^n \lambda _i(X)X_i,~\sum_{i=0}^n\lambda _i(X)\equiv 1.
$$
则称 $\lambda _i(X)$ 为齐次坐标,此外,若 $\lambda _i(X)\ge 0,~i=0,\cdots,n$,则称 $\lambda _i(X)$ 为重心坐标。显然可以将上式看作一个以 $\lambda _i(X)$ 为未知量,$X$ 和 $X_i$ 为已知量的非齐次线性方程组,即重心坐标是方程组的非负解。当 $n=2$ 时,
$$
\begin{cases}
\lambda _0(X)x_0+\lambda _1(X)x_1+\lambda _2(X)x_2=x,\\
\lambda _0(X)y_0+\lambda _1(X)y_1+\lambda _2(X)y_2=y,\\
\lambda _0(X)+\lambda _1(X)+\lambda _2(X)=1.
\end{cases}
$$
即
$$
\begin{bmatrix}
x_0 & x_1 & x_2\\
y_0 & y_1 & y_2\\
1 & 1 & 1\\
\end{bmatrix}
\begin{bmatrix}
\lambda _0(X)\\
\lambda _1(X)\\
\lambda _2(X)\\
\end{bmatrix}=\begin{bmatrix}
x\\
y\\
1\\
\end{bmatrix}.
$$
由 $\begin{bmatrix}
x_0 & x_1 & x_2\\
y_0 & y_1 & y_2\\
1 & 1 & 1\\
\end{bmatrix}$ 的非奇异可知,$\lambda _0(X)$,$\lambda _1(X)$,$\lambda _2(X)$ 存在唯一。

重心坐标的几何意义:

\begin{figure}[H]
\centering
\includegraphics[scale=0.7]{./figures/2.png}
\caption{}
\end{figure}

任取 $P\in \triangle_{X_0 X_1 X_2}$,延长 $X_2 P$ 使之与 $X_0 X_1$ 相交于点 $Q$,则
$$
\lambda _2=\frac {PQ}{X_2 P}=\frac{S_{\triangle PX_0 X_1}}{S_{\triangle X_0 X_1 X_2}}.
$$

类似的可以定义一维,高维的重心坐标,定义高维重心坐标时,顶点要遵循右手法则。

\section{单纯形上的基函数}
区间 $[x_0,x_1]$ 上 $p$ 次拉格朗日基函数的构造

首先构造一维区间 $[x_0,x_1]$ 上的重心坐标,
$$
\begin{cases}
\lambda _0(x)x_0+\lambda _1(x)x_1=x,\\
\lambda _0(x)+\lambda _1(x)=1.
\end{cases}
$$
则,
$$
\begin{bmatrix}
x_0 & x_1\\
1 & 1\\
\end{bmatrix}
\begin{bmatrix}
\lambda _0(x)\\
\lambda _1(x)\\
\end{bmatrix}=\begin{bmatrix}
x\\
1\\
\end{bmatrix},
$$
因此,
$$
\lambda _0(x)=\frac{\begin{vmatrix}
x & x_1 \\
1 & 1 \\
\end{vmatrix}}{\begin{vmatrix}
x_0 & x_1 \\
1 & 1 \\
\end{vmatrix}}=\frac{x-x_1}{x_0-x_1}=\frac{x_1-x_0}{x_1-x_0},
\quad\lambda _1(x)=\frac{\begin{vmatrix}
x_0 & x \\
1 & 1
\end{vmatrix}}{\begin{vmatrix}
x_0 & x_1 \\
1 & 1
\end{vmatrix}}=\frac{x_0-x}{x_0-x_1}=\frac{x-x_0}{x_1-x_0}.
$$
$\forall x\in [x_0,x_1]$,则 $\lambda _0(x)\ge 0$,$\lambda _1(x)\ge 0$.并且
$$
\begin{cases}
\lambda _0 (x_0)=1,\lambda _0 (x_1)=0,\\
\lambda _1 (x_0)=0,\lambda _1 (x_1)=1.\\
\end{cases}
$$
易知, $\lambda_0, \lambda_1$ 都是关于 $x$ 的线性函数(这里指一次函数)。

重心坐标关于 $x$ 的导数为:
$$
\frac{\mathrm d \lambda_0}{\mathrm dx} = -\frac{1}{x_1 - x_0},\quad 
\frac{\mathrm d \lambda_1}{\mathrm dx} = \frac{1}{x_1 - x_0}.
$$
区间 $[x_0, x_1]$ 上的 $p\geq 1$ 次基函数共有 
$n_{dof} = p+1$ 个,

其插值基函数的计算公式如下:
$$
\phi_{m,n} = \frac{p^p}{m!n!}\prod_{l_0 = 0}^{m - 1}
(\lambda_0 - \frac{l_0}{p}) \prod_{l_1 = 0}^{n-1}(\lambda_1 -
\frac{l_1}{p}).
$$
其中 $m\geq 0$, $n\geq 0$, 且 $m+n=p$,即 $\phi_{m,n}$ 是一个 $m+n=p$ 次多项式。

由 $\frac{x_1 - x}{x_1 - x_0}=\frac{l_0}{p}$ 得到
$$
x=x_1-\frac{l_0}{p} (x_1 - x_0)=(1-\frac{l_0}{p})x_1 + \frac{l_0}{p} x_0.
$$
因为 $(1-\frac{l_0}{p}) + \frac{l_0}{p} =1$,故 $x\in [x_0,x_1].$

由 $\frac{x - x_0}{x_1 - x_0}=\frac{l_1}{p}$ 得到
$$
x=x_0+\frac{l_1}{p} (x_1 - x_0)=\frac{l_1}{p})x_1 + (1-\frac{l_1}{p})x_0.
$$
因为 $(1-\frac{l_1}{p}) + \frac{l_1}{p} = 1$,故 $x\in [x_0,x_1].$

因此插值基函数过节点 $x_1,\frac{(p-1)x_1 + x_0}{p},\frac{(p-2)x_1 + 2x_0}{p},\frac{(p-3)x_1 + 3x_0}{p},\cdots,x_0$,即把区间 $[x_0,x_1]$ 均匀等分,步长为 $\frac{(x_1-x_0)}{p}$.

这个 $p$ 次插值基函数实际上就是这种形式的
$$
l_k(x)=\frac{(x-x_0)\cdots(x-x_{k-1})(x-x_{k+1})\cdots(x - x_p)}{(x_k-x_0)\cdots(x_k-x_{k-1})(x_k-x_{k+1})\cdots(x_k-x_p)}
$$
拉格朗日插值基函数。

$p$ 次基函数的面向数组的计算
构造向量:
$$
P = ( \frac{1}{0!},  \frac{1}{1!}, \frac{1}{2!}, \cdots, \frac{1}{p!})
$$

构造矩阵:
$$
A :=                                                                            
\begin{pmatrix}  
1  &  1  \\
\lambda_0 & \lambda_1\\                                             
\lambda_0 - \frac{1}{p} & \lambda_1 - \frac{1}{p}\\   
\vdots & \vdots \\                                                     
\lambda_0 - \frac{p - 1}{p} & \lambda_1 - \frac{p - 1}{p}
\end{pmatrix}                                                                   
$$ 
对 $A$ 的每一列做累乘运算, 并左乘由 $P$ 形成的对角矩阵, 得矩阵:

$$
B = \mathrm{diag}(P)
\begin{pmatrix}
1 & 1\\
\lambda_0 & \lambda_1\\
\prod_{l=0}^{1}(\lambda_0 - \frac{l}{p}) & \prod_{l=0}^{1}(\lambda_1 - \frac{l}{p})\\
\vdots & \vdots \\
\prod_{l=0}^{p-1}(\lambda_0 - \frac{l}{p}) & \prod_{l=0}^{p-1}(\lambda_1 - \frac{l}{p}) 
\end{pmatrix}
$$

易知, 只需从 $B$ 的每一列中各选择一项相乘(要求二项次数之和为 $p$), 再乘以 $p^p$ 即可得到相应的基函数, 其中取法共有 

$$
n_{dof} = {p+1}
$$

构造指标矩阵:

$$
I = \begin{pmatrix}
p  & 0 \\ p-1 & 1 \\ \vdots & \vdots \\ 0 & p 
\end{pmatrix}
$$
由$I$可知一共有$\begin{matrix} p+1 \\ \overbrace{ 1+1+\cdots+1 }
\end{matrix}$种选择。

\section{三角形单元}
下面考虑二维三角形单元的重心坐标。

给定三角形单元 $\tau$, 其三个顶点 $\mathbf x_0 :=(x_0,y_0)$, $\mathbf x_1 :=(x_1,y_1)$ 和 $\mathbf x_2 :=(x_2,y_2)$ 逆时针排列, 且不在同一条直线上, 那么向量 $\overrightarrow{\mathbf x_0\mathbf x_1}$ 和 $\overrightarrow{\mathbf x_0\mathbf x_2}$ 是线性无关的. 这等价于矩阵

$$
A = 
\begin{pmatrix}
x_0 & x_1 & x_2 \\
y_0 & y_1 & y_2 \\
1   & 1   & 1 
\end{pmatrix}
$$

非奇异. 
又 $|A|\ne 0$,故任给一点 $\mathbf{x}:=(x,y)\in\tau$,下面的线性方程组

$$
A 
\begin{pmatrix}
\lambda_0 \\
\lambda_1\\
\lambda_2  
\end{pmatrix}
=\begin{pmatrix}
x \\
y\\
1  
\end{pmatrix}
$$

有唯一的一组解 $\lambda_0,\lambda_1,\lambda_2$. 
因此对任意二维点 $\mathbf{x}\in\tau$, 有
$$
\begin{cases}
\lambda _0 x_0+\lambda _1 x_1+\lambda _2 x_2=x,\\
\lambda _0 y_0+\lambda _1 y_1+\lambda _2 y_2=y,\\
\lambda _0+ \lambda _1 +\lambda _2 =1.
\end{cases}
$$
故$\lambda _0 (x_0,y_0)+\lambda _1 (x_1,y_1)+\lambda _2 (x_2,y_2)=(x,y)$,
$$
\mathbf{x}=\lambda_0(\mathbf{x})\mathbf{x}_0 + \lambda_1(\mathbf{x})\mathbf{x}_1 + \lambda_2(\mathbf{x})\mathbf{x}_2 
\text{ 且 } \lambda_0(\mathbf{x}) + \lambda_1(\mathbf{x}) + \lambda_2(\mathbf{x}) = 1. 
$$

$\lambda_0,\lambda_1,\lambda_2$ 称为点 $\mathbf{x}$ 关于点 $\mathbf{x}_0,\mathbf{x}_1$ 和$\mathbf{x}_2$ 的重心坐标. 

易知, $\lambda_0, \lambda_1, \lambda_2$ 都是关于 $\mathbf x$ 的线性函数, 且有

\begin{eqnarray*}
\lambda_0(\mathbf x_0) = 1,\quad & \lambda_0(\mathbf x_1) = 0,\quad& \lambda_0(\mathbf x_2) = 0\\
\lambda_1(\mathbf x_0) = 0,\quad & \lambda_1(\mathbf x_1) = 1,\quad& \lambda_1(\mathbf x_2) = 0\\
\lambda_2(\mathbf x_0) = 0,\quad & \lambda_2(\mathbf x_1) = 0,\quad & \lambda_2(\mathbf x_2) = 1\\
\end{eqnarray*}

\begin{figure}[H]
\centering
\includegraphics[scale=0.7]{./figures/3.png}
\caption{}
\end{figure}

也可以通过下面的方式计算重心坐标。

设 $\lambda_0 (x,y)=ax+by+c$,则
$$
\begin{cases}
\lambda _0 (x_0,y_0)=ax_0+by_0+c=1,\\
\lambda _0 (x_1,y_1)=ax_1+by_1+c=0,\\
\lambda _0 (x_2,y_2)=ax_2+by_2+c=0.\\
\end{cases}
$$
从而可以求出 $a$,$b$,$c$,进而求出 $\lambda _0$.同理可得 $\lambda _1$,$\lambda _2$.

为了考虑变化率的大小和方向,引入梯度。梯度的大小是变化率最大的值,梯度的方向是变化率最大的方向。

$\lambda_0, \lambda_1, \lambda_2$ 关于 $\mathbf x$ 的梯度分别为:

$$
\begin{aligned}
\nabla\lambda_0 = \frac{1}{2|\tau|}(\mathbf x_2 - \mathbf x_1)W\\
\nabla\lambda_1 = \frac{1}{2|\tau|}(\mathbf x_0 - \mathbf x_2)W\\
\nabla\lambda_2 = \frac{1}{2|\tau|}(\mathbf x_1 - \mathbf x_0)W\\
\end{aligned}
$$

其中 

$$
W = \begin{pmatrix}
0 & 1\\ -1 & 0 
\end{pmatrix}
$$
三角形单元上的 $p$ 次基函数公式

给定三角形单元上的一个重心坐标 $(\lambda_0, \lambda_1, \lambda_2)$, 所有 $p\geq 1$ 次基函数的计算公式如下:

$$
\phi_{m,n,k} = \frac{p^p}{m!n!k!}\prod_{l_0 = 0}^{m - 1}
(\lambda_0 - \frac{l_0}{p}) \prod_{l_1 = 0}^{n-1}(\lambda_1 -
\frac{l_1}{p}) \prod_{l_2=0}^{k-1}(\lambda_2 - \frac{l_2}{p}).
$$

其中 $ m\geq 0$, $n\geq 0$, $ k \geq 0$, 且 $m+n+k=p$.

因为$\lambda_0 (x_2,y_2) = 0,\lambda_1 (x_0,y_0) = 0,\lambda_2 (x_1,y_1) = 0$,因此插值基函数一定过节点$(x_0,y_0),(x_1,y_1),(x_2,y_2).$
因为$\lambda_0 (\frac{x_1+x_2}{2}) = 0,\lambda_1 (\frac{x_0+x_2}{2}) = 0,\lambda_2 (\frac{x_0+x_1}{2}) = 0$,因此插值基函数过节点$(\frac{x_1+x_2}{2},\frac{y_1+y_2}{2}),(\frac{x_0+x_2}{2},\frac{y_0+y_2}{2}),(\frac{x_0+x_1}{2},\frac{y_0+y_1}{2})$

$\cdots$
即把三角形单元均匀剖分。




















\newpage
\nocite{*}
\bibliography{ref}
\end{document}

