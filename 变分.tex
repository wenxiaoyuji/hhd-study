% !Mode:: "TeX:UTF-8"
\documentclass{article}
\input{en_preamble.tex}
\input{xecjk_preamble.tex}
\setCJKmainfont{STKaiti} % 如果请替换为本地系统有的字体
%中文断行
\XeTeXlinebreaklocale "zh"
\XeTeXlinebreakskip = 0pt plus 1pt minus 0.1pt
\begin{document}
%\title{变分}
%\date{\today}
%\maketitle  %定义时间和标题
\tableofcontents
%\newpage
\section{定义}
\subsection{函数的变分}
函数$y(x)$的变分定义为$\delta y=y_1(x)-y(x)$,其中$y_1(x)$是“靠近”$y(x)$的一个函数,即$\delta y$是同一自变量$x$处相邻函数的函数值之差。

注意:

$(\delta y)'=y'_1(x)-y'(x)=\delta y'$

$(\delta y)^n=\delta y^n $
\subsection{泛函的变分}
定义泛函$J[y(x)]=\int_{a}^{b} f(x,y,y')\mathrm{d}x$,则增量$\bigtriangleup J=\int_{a}^{b}[f(x,y+\delta y,y'+\delta y')-f(x,y,y')]\mathrm{d}x=\int_{a}^{b}[\frac{\partial f}{\partial y}\delta y + \frac{\partial f}{\partial y'}\delta y'+\frac{1}{2}\frac{\partial^2 f}{\partial^2 y}(\delta y)^2+\frac{\partial^2 f}{\partial y\,\partial y'}\delta y\delta y'+\frac{1}{2}\frac{\partial^2 f}{\partial y'^2}(\partial y')^2+\cdots]\mathrm{d}x$

舍弃掉$\delta y$和$\delta y'$二次项及以上的高次项,得到关于$\delta y$和$\delta y'$一次项的和,称为$J[y(X)]=\int_{a}^{b} f(x,y,y')\mathrm{d}x$的一阶变分,简称为泛函的变分,即$\delta J=\int_{a}^{b}(\frac{\partial f}{\partial y}\delta y + \frac{\partial f}{\partial y'}\delta y')\mathrm{d}x$。
\subsection{泛函变分的基本运算法则}
泛函变分运算法则与微分运算法则基本相同

$(\delta F_1 +F_2)=\delta F_1 +\delta F_2$

$(\delta F)^n=nF^n-1\delta F$

$(\delta F_1 \cdot F_2)=F_2\delta F_1+F_1\delta F_2$

$(\delta (\frac{F_1}{F_2})=\frac{1}{F^2_2}(F_2\delta F_1-F_1\delta F_2)$

$\delta\int_{a}^{b}F\mathrm{d}x=\int_{a}^{b}\delta F\mathrm{d}x$







%如果将泛函取极值时的函数定义为$y(x)$,并且定义与函数$y(x)$相靠近的函数为$y(x,\varepsilon)$,记为$y(x,\varepsilon)=y(x)+\varepsilon\eta(x)$,$\varepsilon$是一个参数。
%函数$y(x)$的变分定义为$\delta y=\eta(x)\mathrm{d}\varepsilon$,由此可得$\delta y'=\eta ^\prime(x)\mathrm{d}\varepsilon$
%定义泛函$J[y(X)]=\int_{a}^{b} F(x,y,y')\mathrm{d}x$的变分为$\delta J=\int_{a}^{b}(\frac{\partial F}{\partial y}\delta y + \frac{\partial F}{\partial y'}\delta y')\mathrm{d}x$。

\subsection{泛函变分举例}
计算泛函$J[y(x)]=\int_{-1}^{1} (y'e^7 +xy^2)\mathrm{d}x$的变分


$\delta J[y(x)]=\delta\int_{-1}^{1}(y'e^7+xy^2)\mathrm{d}x=\int_{-1}^{1}(2xy\delta y+e^7\delta y')\mathrm{d}x=\int_{-1}^{1}(2xy\delta y)\mathrm{d}x+\int_{-1}^{1}e^7\mathrm{d}\delta y=\int_{-1}^{1}(2xy\delta y)\mathrm{d}x
$,最后一步利用上边界上函数变分为0。






%\begin{equation}
%-\nabla \cdot (\beta\nabla u) = f(x,y),\,\ (x,y)\in \Omega
%\end{equation}
%Dilichlet 边界条件

%\begin{equation}
%u(x,y) = g(x,y),\,\ (x,y)\in \partial \Omega
%\end{equation}
%\subsection{符号}
%\begin{tabular}{ |l|l| }   
%\hline   
%\multicolumn{2}{|c|}{符号说明} \\   
%\hline
%符号 & 含义\\
%\hline
%$\Omega$ & 二维长方形区域 \\
%\hline
%$nx$ & $x$ 方向剖分的段数 \\
%\hline
%$ny$ & $y$ 方向剖分的段数 \\
%\hline
%$hx$ &  $x$ 方向每段的长度\\
%\hline
%$hy$ &  $y$ 方向每段的长度 \\
%\hline
%$\mu$ & $the \,\ viscosity \,\ coefficient$ \\
%\hline
%$k$ & $the \,\ permeability \,\ tensor$ \\
%\hline 
%$NC$ & 代表 $cell$ 的个数 \\
%\hline
%$NE$ & 代表总的 $edge$ 的个数 \\
%\hline
%\end{tabular}

\section{欧拉—拉格朗日方程}
欧拉—拉格朗日方程是泛函极值问题的必要条件,假设$J[y(x)]$的极值问题的解为$y=y(x)$,现在推导这个解所满足的微分方程。

$\delta J=\int_{a}^{b}(\frac{\partial f}{\partial y}\delta y + \frac{\partial f}{\partial y'}\delta y')\mathrm{d}x=0$,将第二项分部积分得到$\int_{a}^{b}(\frac{\partial f}{\partial y'}\delta y')\mathrm{d}x=\int_{a}^{b}\frac{\partial f}{\partial y'}\mathrm{d}\delta y$,因为$\delta y(a)=0$,$\delta y(b)=0$,所以$\int_{a}^{b}(\frac{\partial f}{\partial y'}\delta y')\mathrm{d}x=-\int_{a}^{b}\delta y\mathrm{d}\frac{\partial f}{\partial y'}$,因此$\delta J=\int_{a}^{b}\frac{\partial f}{\partial y}\delta y\mathrm{d}x-\int_{a}^{b}\delta y\mathrm{d}\frac{\partial f}{\partial y'}=\int_{a}^{b}(\frac{\partial f}{\partial y}-\frac{\mathrm{d}\frac{\partial f}{\partial y'}}{\mathrm{d}x})\delta y\mathrm{d}x=0$,因为对于任何函数$\delta y$都成立,故$\frac{\partial f}{\partial y}-\frac{\mathrm{d}\frac{\partial f}{\partial y'}}{\mathrm{d}x}=0$,这就是欧拉—拉格朗日方程。
%\begin{equation*}
%\begin{cases}
%\begin{aligned}
%\frac{\mu}{k}\mathbf{u} + \nabla p & = 0 \quad in \,\ \Omega = (0,1)\times (0,1) \\
%\nabla \cdot \mathbf{u} & = f \quad in \,\ \Omega \\
%\mathbf {u} & = 0 \quad on \,\ \partial \Omega
%\end{aligned}
%\end{cases}
%\end{equation*}

%且有 \\
%\begin{equation*}
%\int_{\Omega}f dxdy = 0
%\end{equation*}

%记 $u$ 为 $\mathbf{u}$ 在 $x$ 方向的分量,$v$ 为 $\mathbf{u}$ 在 $y$ 方向的分量,则有 \\

%\begin{equation*}
%\begin{cases}
%\begin{aligned}
%\frac{\mu}{k}\cdot u + \partial_x p & = 0 \quad (1) \\
%\frac{\mu}{k}\cdot v + \partial_y p & = 0 \quad (2) \\
%\partial_x u + \partial_y v & = f \quad (3)
%\end{aligned}
%\end{cases}
%\end{equation*}

%\section{离散后组装矩阵}
%利用一阶向前差分把方程变成差分方程,现在从 $edge$ 和 $cell$ 的角度考虑模型。 \\

%对于 $(1)$, 从内部纵向 $edge$ 的角度考虑:
%我们需要找到内部纵向 $edge$ 所对应的左手边的 $cell$ 和右手边的 $cell$. 左右两边的$cell$ 所对应的 $p$ 分别记为 $p_{l}$、$p_{r}$.$u$ 为 $edge$ 的中点,记为 $u_m$。按照 $mesh$ 里的编号规则排序。\\

%则每条内部边上所对应的差分方程为:

%\begin{equation*}
%\frac{\mu}{k} \cdot u_m + \frac{p_r - p_l}{hx} = 0
%\end{equation*}

%对于 $(2)$,从内部横向 $edge$ 的角度考虑:
%我们需要找到内部横向 $edge$ 所对应的左手边的 $cell$ 和右手边的 $cell$. $cell$ 所对应的 $p$ 与 $(1)$ 中的相同。$v$ 为 $edge$ 的中点,记为 $v_m$。\\

%则每条内部边上所对应的差分方程为:\\

%\begin{equation*}
%\frac{\mu}{k} \cdot v_m + \frac{p_l - p_r}{hy} = 0
%\end{equation*}

%对于 $(3)$, 从 $cell$ 的角度考虑:
%由于单元是四边形单元,我们记单元所对应边的局部编号为[0,1,2,3](StructureQuadMesh.py 里的网格),第 $i$ 个单元所对应的边记为 $e_{i,0},e_{i,1},e_{i,2},e_{i,3}$。\\

%则 $(3)$ 式第 $i$ 个单元所对应的差分方程为:\\

%\begin{equation*}
%\frac{u_{e_{i,1}} - u_{e_{i,3}}}{hx} + \frac{v_{e_{i,2}} - v_{e_{i,0}}}{hy} = f_i
%\end{equation*}

%我们需要生成一个 $(NE+NC)\times(NE+NC)$的系数矩阵,把它看成分块矩阵
%\begin{equation*}
%\begin{pmatrix}
% A_{1,1} & A_{1,2} \\
%A_{2,1} & A_{2,2}
%\end{pmatrix}
%\end{equation*}

%其中 \\

%\begin{equation*}
%\begin{aligned}
%A_{1,1} : NE\times NE \\
%A_{1,2} : NE\times NC \\
%A_{2,1} : NC\times NE \\
%A_{2,2} : NC\times NC
%\end{aligned}
%\end{equation*}

%$A_{1,1}$ 对应的是 $(1),(2)$ 两式的第一项,即含有 $u,v$ 的项,$A_12$ 对应的是 $(1),(2)$ 两式的第二项。

%\newpage
%\nocite{*}
%\bibliography{ref}
\end{document}

