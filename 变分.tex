% !Mode:: "TeX:UTF-8"
\documentclass{article}

%%%%%%%%------------------------------------------------------------------------
%%%% 日常所用宏包

%% 控制页边距
% 如果是beamer文档类, 则不用geometry
\makeatletter
\@ifclassloaded{beamer}{}{\usepackage[top=2.5cm, bottom=2.5cm, left=2.5cm, right=2.5cm]{geometry}}
\makeatother

%% 控制项目列表
\usepackage{enumerate}

%% 多栏显示
\usepackage{multicol}

%% 算法环境
\usepackage{algorithm}  
\usepackage{algorithmic} 
\usepackage{float} 

%% 网址引用
\usepackage{url}

%% 控制矩阵行距
\renewcommand\arraystretch{1.4}

%% hyperref宏包,生成可定位点击的超链接,并且会生成pdf书签
\makeatletter
\@ifclassloaded{beamer}{
\usepackage{hyperref}
\usepackage{ragged2e} % 对齐
}{
\usepackage[%
    pdfstartview=FitH,%
    CJKbookmarks=true,%
    bookmarks=true,%
    bookmarksnumbered=true,%
    bookmarksopen=true,%
    colorlinks=true,%
    citecolor=blue,%
    linkcolor=blue,%
    anchorcolor=green,%
    urlcolor=blue%
]{hyperref}
}
\makeatother



\makeatletter % 如果是 beamer 不需要下面两个包
\@ifclassloaded{beamer}{
\mode<presentation>
{
} 
}{
%% 控制标题
\usepackage{titlesec}
%% 控制目录
\usepackage{titletoc}
}
\makeatother

%% 控制表格样式
\usepackage{booktabs}

%% 控制字体大小
\usepackage{type1cm}

%% 首行缩进,用\noindent取消某段缩进
\usepackage{indentfirst}

%% 支持彩色文本、底色、文本框等
\usepackage{color,xcolor}

%% AMS LaTeX宏包: http://zzg34b.w3.c361.com/package/maths.htm#amssymb
\usepackage{amsmath,amssymb}
%% 多个图形并排
\usepackage{subfig}
%%%% 基本插图方法
%% 图形宏包
\usepackage{graphicx}
\newcommand{\red}[1]{\textcolor{red}{#1}}
\newcommand{\blue}[1]{\structure{#1}}
\newcommand{\brown}[1]{\textcolor{brown}{#1}}
\newcommand{\green}[1]{\textcolor{green}{#1}}


%%%% 基本插图方法结束

%%%% pgf/tikz绘图宏包设置
\usepackage{pgf,tikz}
\usetikzlibrary{shapes,automata,snakes,backgrounds,arrows}
\usetikzlibrary{mindmap}
%% 可以直接在latex文档中使用graphviz/dot语言,
%% 也可以用dot2tex工具将dot文件转换成tex文件再include进来
%% \usepackage[shell,pgf,outputdir={docgraphs/}]{dot2texi}
%%%% pgf/tikz设置结束


\makeatletter % 如果是 beamer 不需要下面两个包
\@ifclassloaded{beamer}{

}{
%%%% fancyhdr设置页眉页脚
%% 页眉页脚宏包
\usepackage{fancyhdr}
%% 页眉页脚风格
\pagestyle{plain}
}

%% 有时会出现\headheight too small的warning
\setlength{\headheight}{15pt}

%% 清空当前页眉页脚的默认设置
%\fancyhf{}
%%%% fancyhdr设置结束


\makeatletter % 对 beamer 要重新设置
\@ifclassloaded{beamer}{

}{
%%%% 设置listings宏包用来粘贴源代码
%% 方便粘贴源代码,部分代码高亮功能
\usepackage{listings}

%% 设置listings宏包的一些全局样式
%% 参考http://hi.baidu.com/shawpinlee/blog/item/9ec431cbae28e41cbe09e6e4.html
\lstset{
showstringspaces=false,              %% 设定是否显示代码之间的空格符号
numbers=left,                        %% 在左边显示行号
numberstyle=\tiny,                   %% 设定行号字体的大小
basicstyle=\footnotesize,                    %% 设定字体大小\tiny, \small, \Large等等
keywordstyle=\color{blue!70}, commentstyle=\color{red!50!green!50!blue!50},
                                     %% 关键字高亮
frame=shadowbox,                     %% 给代码加框
rulesepcolor=\color{red!20!green!20!blue!20},
escapechar=`,                        %% 中文逃逸字符,用于中英混排
xleftmargin=2em,xrightmargin=2em, aboveskip=1em,
breaklines,                          %% 这条命令可以让LaTeX自动将长的代码行换行排版
extendedchars=false                  %% 这一条命令可以解决代码跨页时,章节标题,页眉等汉字不显示的问题
}}
\makeatother
%%%% listings宏包设置结束


%%%% 附录设置
\makeatletter % 对 beamer 要重新设置
\@ifclassloaded{beamer}{

}{
\usepackage[title,titletoc,header]{appendix}
}
\makeatother
%%%% 附录设置结束


%%%% 日常宏包设置结束
%%%%%%%%------------------------------------------------------------------------


%%%%%%%%------------------------------------------------------------------------
%%%% 英文字体设置结束
%% 这里可以加入自己的英文字体设置
%%%%%%%%------------------------------------------------------------------------

%%%%%%%%------------------------------------------------------------------------
%%%% 设置常用字体字号,与MS Word相对应

%% 一号, 1.4倍行距
\newcommand{\yihao}{\fontsize{26pt}{36pt}\selectfont}
%% 二号, 1.25倍行距
\newcommand{\erhao}{\fontsize{22pt}{28pt}\selectfont}
%% 小二, 单倍行距
\newcommand{\xiaoer}{\fontsize{18pt}{18pt}\selectfont}
%% 三号, 1.5倍行距
\newcommand{\sanhao}{\fontsize{16pt}{24pt}\selectfont}
%% 小三, 1.5倍行距
\newcommand{\xiaosan}{\fontsize{15pt}{22pt}\selectfont}
%% 四号, 1.5倍行距
\newcommand{\sihao}{\fontsize{14pt}{21pt}\selectfont}
%% 半四, 1.5倍行距
\newcommand{\bansi}{\fontsize{13pt}{19.5pt}\selectfont}
%% 小四, 1.5倍行距
\newcommand{\xiaosi}{\fontsize{12pt}{18pt}\selectfont}
%% 大五, 单倍行距
\newcommand{\dawu}{\fontsize{11pt}{11pt}\selectfont}
%% 五号, 单倍行距
\newcommand{\wuhao}{\fontsize{10.5pt}{10.5pt}\selectfont}
%%%%%%%%------------------------------------------------------------------------


%% 设定段间距
\setlength{\parskip}{0.5\baselineskip}

%% 设定行距
\linespread{1}


%% 设定正文字体大小
% \renewcommand{\normalsize}{\sihao}

%制作水印
\RequirePackage{draftcopy}
\draftcopyName{XTUMESH}{100}
\draftcopySetGrey{0.90}
\draftcopyPageTransform{40 rotate}
\draftcopyPageX{350}
\draftcopyPageY{80}

%%%% 个性设置结束
%%%%%%%%------------------------------------------------------------------------


%%%%%%%%------------------------------------------------------------------------
%%%% bibtex设置

%% 设定参考文献显示风格
% 下面是几种常见的样式
% * plain: 按字母的顺序排列,比较次序为作者、年度和标题
% * unsrt: 样式同plain,只是按照引用的先后排序
% * alpha: 用作者名首字母+年份后两位作标号,以字母顺序排序
% * abbrv: 类似plain,将月份全拼改为缩写,更显紧凑
% * apalike: 美国心理学学会期刊样式, 引用样式 [Tailper and Zang, 2006]

\makeatletter
\@ifclassloaded{beamer}{
\bibliographystyle{apalike}
}{
\bibliographystyle{unsrt}
}
\makeatother


%%%% bibtex设置结束
%%%%%%%%------------------------------------------------------------------------

%%%%%%%%------------------------------------------------------------------------
%%%% xeCJK相关宏包

\usepackage{xltxtra,fontspec,xunicode}
\usepackage[slantfont, boldfont]{xeCJK} 

%% 针对中文进行断行
\XeTeXlinebreaklocale "zh"             

%% 给予TeX断行一定自由度
\XeTeXlinebreakskip = 0pt plus 1pt minus 0.1pt

%%%% xeCJK设置结束                                       
%%%%%%%%------------------------------------------------------------------------

%%%%%%%%------------------------------------------------------------------------
%%%% xeCJK字体设置

%% 设置中文标点样式,支持quanjiao、banjiao、kaiming等多种方式
\punctstyle{kaiming}                                        
                                                     
%% 设置缺省中文字体
\setCJKmainfont[BoldFont={Adobe Heiti Std}, ItalicFont={Adobe Kaiti Std}]{Adobe Song Std}   
%% 设置中文无衬线字体
\setCJKsansfont[BoldFont={Adobe Heiti Std}]{Adobe Kaiti Std}  
%% 设置等宽字体
\setCJKmonofont{Adobe Heiti Std}                            

%% 英文衬线字体
\setmainfont{DejaVu Serif}                                  
%% 英文等宽字体
\setmonofont{DejaVu Sans Mono}                              
%% 英文无衬线字体
\setsansfont{DejaVu Sans}                                   

%% 定义新字体
\setCJKfamilyfont{song}{Adobe Song Std}                     
\setCJKfamilyfont{kai}{Adobe Kaiti Std}
\setCJKfamilyfont{hei}{Adobe Heiti Std}
\setCJKfamilyfont{fangsong}{Adobe Fangsong Std}
\setCJKfamilyfont{lisu}{LiSu}
\setCJKfamilyfont{youyuan}{YouYuan}

%% 自定义宋体
\newcommand{\song}{\CJKfamily{song}}                       
%% 自定义楷体
\newcommand{\kai}{\CJKfamily{kai}}                         
%% 自定义黑体
\newcommand{\hei}{\CJKfamily{hei}}                         
%% 自定义仿宋体
\newcommand{\fangsong}{\CJKfamily{fangsong}}               
%% 自定义隶书
\newcommand{\lisu}{\CJKfamily{lisu}}                       
%% 自定义幼圆
\newcommand{\youyuan}{\CJKfamily{youyuan}}                 

%%%% xeCJK字体设置结束
%%%%%%%%------------------------------------------------------------------------

%%%%%%%%------------------------------------------------------------------------
%%%% 一些关于中文文档的重定义
\newcommand{\chntoday}{\number\year\,年\,\number\month\,月\,\number\day\,日}
%% 数学公式定理的重定义

%% 中文破折号,据说来自清华模板
\newcommand{\pozhehao}{\kern0.3ex\rule[0.8ex]{2em}{0.1ex}\kern0.3ex}

\newtheorem{example}{例}                                   
\newtheorem{theorem}{定理}[section]                         
\newtheorem{definition}{定义}
\newtheorem{axiom}{公理}
\newtheorem{property}{性质}
\newtheorem{proposition}{命题}
\newtheorem{lemma}{引理}
\newtheorem{corollary}{推论}
\newtheorem{remark}{注解}
\newtheorem{condition}{条件}
\newtheorem{conclusion}{结论}
\newtheorem{assumption}{假设}

\makeatletter %
\@ifclassloaded{beamer}{

}{
%% 章节等名称重定义
\renewcommand{\contentsname}{目录}     
\renewcommand{\indexname}{索引}
\renewcommand{\listfigurename}{插图目录}
\renewcommand{\listtablename}{表格目录}
\renewcommand{\appendixname}{附录}
\renewcommand{\appendixpagename}{附录}
\renewcommand{\appendixtocname}{附录}
%% 设置chapter、section与subsection的格式
\titleformat{\chapter}{\centering\huge}{第\thechapter{}章}{1em}{\textbf}
\titleformat{\section}{\centering\sihao}{\thesection}{1em}{\textbf}
\titleformat{\subsection}{\xiaosi}{\thesubsection}{1em}{\textbf}
\titleformat{\subsubsection}{\xiaosi}{\thesubsubsection}{1em}{\textbf}

\@ifclassloaded{book}{

}{
\renewcommand{\abstractname}{摘要}
}
}
\makeatother

\renewcommand{\figurename}{图}
\renewcommand{\tablename}{表}

\makeatletter
\@ifclassloaded{book}{
\renewcommand{\bibname}{参考文献}
}{
\renewcommand{\refname}{参考文献} 
}
\makeatother

\floatname{algorithm}{算法}
\renewcommand{\algorithmicrequire}{\textbf{输入:}}
\renewcommand{\algorithmicensure}{\textbf{输出:}}

%%%% 中文重定义结束
%%%%%%%%------------------------------------------------------------------------

\setCJKmainfont{STKaiti} % 如果请替换为本地系统有的字体
%中文断行
\XeTeXlinebreaklocale "zh"
\XeTeXlinebreakskip = 0pt plus 1pt minus 0.1pt
\begin{document}
%\title{变分}
%\date{\today}
%\maketitle  %定义时间和标题
\tableofcontents
%\newpage
\section{定义}
\subsection{函数的变分}
函数$y(x)$的变分定义为$\delta y=y_1(x)-y(x)$,其中$y_1(x)$是“靠近”$y(x)$的一个函数,即$\delta y$是同一自变量$x$处相邻函数的函数值之差。

注意:

$(\delta y)'=y'_1(x)-y'(x)=\delta y'$

$(\delta y)^n=\delta y^n $
\subsection{泛函的变分}
定义泛函$J[y(x)]=\int_{a}^{b} f(x,y,y')\mathrm{d}x$,则增量$\bigtriangleup J=\int_{a}^{b}[f(x,y+\delta y,y'+\delta y')-f(x,y,y')]\mathrm{d}x=\int_{a}^{b}[\frac{\partial f}{\partial y}\delta y + \frac{\partial f}{\partial y'}\delta y'+\frac{1}{2}\frac{\partial^2 f}{\partial^2 y}(\delta y)^2+\frac{\partial^2 f}{\partial y\,\partial y'}\delta y\delta y'+\frac{1}{2}\frac{\partial^2 f}{\partial y'^2}(\partial y')^2+\cdots]\mathrm{d}x$

舍弃掉$\delta y$和$\delta y'$二次项及以上的高次项,得到关于$\delta y$和$\delta y'$一次项的和,称为$J[y(X)]=\int_{a}^{b} f(x,y,y')\mathrm{d}x$的一阶变分,简称为泛函的变分,即$\delta J=\int_{a}^{b}(\frac{\partial f}{\partial y}\delta y + \frac{\partial f}{\partial y'}\delta y')\mathrm{d}x$。
\subsection{泛函变分的基本运算法则}
泛函变分运算法则与微分运算法则基本相同

$(\delta F_1 +F_2)=\delta F_1 +\delta F_2$

$(\delta F)^n=nF^n-1\delta F$

$(\delta F_1 \cdot F_2)=F_2\delta F_1+F_1\delta F_2$

$(\delta (\frac{F_1}{F_2})=\frac{1}{F^2_2}(F_2\delta F_1-F_1\delta F_2)$

$\delta\int_{a}^{b}F\mathrm{d}x=\int_{a}^{b}\delta F\mathrm{d}x$







%如果将泛函取极值时的函数定义为$y(x)$,并且定义与函数$y(x)$相靠近的函数为$y(x,\varepsilon)$,记为$y(x,\varepsilon)=y(x)+\varepsilon\eta(x)$,$\varepsilon$是一个参数。
%函数$y(x)$的变分定义为$\delta y=\eta(x)\mathrm{d}\varepsilon$,由此可得$\delta y'=\eta ^\prime(x)\mathrm{d}\varepsilon$
%定义泛函$J[y(X)]=\int_{a}^{b} F(x,y,y')\mathrm{d}x$的变分为$\delta J=\int_{a}^{b}(\frac{\partial F}{\partial y}\delta y + \frac{\partial F}{\partial y'}\delta y')\mathrm{d}x$。

\subsection{泛函变分举例}
计算泛函$J[y(x)]=\int_{-1}^{1} (y'e^7 +xy^2)\mathrm{d}x$的变分


$\delta J[y(x)]=\delta\int_{-1}^{1}(y'e^7+xy^2)\mathrm{d}x=\int_{-1}^{1}(2xy\delta y+e^7\delta y')\mathrm{d}x=\int_{-1}^{1}(2xy\delta y)\mathrm{d}x+\int_{-1}^{1}e^7\mathrm{d}\delta y=\int_{-1}^{1}(2xy\delta y)\mathrm{d}x
$,最后一步利用上边界上函数变分为0。






%\begin{equation}
%-\nabla \cdot (\beta\nabla u) = f(x,y),\,\ (x,y)\in \Omega
%\end{equation}
%Dilichlet 边界条件

%\begin{equation}
%u(x,y) = g(x,y),\,\ (x,y)\in \partial \Omega
%\end{equation}
%\subsection{符号}
%\begin{tabular}{ |l|l| }   
%\hline   
%\multicolumn{2}{|c|}{符号说明} \\   
%\hline
%符号 & 含义\\
%\hline
%$\Omega$ & 二维长方形区域 \\
%\hline
%$nx$ & $x$ 方向剖分的段数 \\
%\hline
%$ny$ & $y$ 方向剖分的段数 \\
%\hline
%$hx$ &  $x$ 方向每段的长度\\
%\hline
%$hy$ &  $y$ 方向每段的长度 \\
%\hline
%$\mu$ & $the \,\ viscosity \,\ coefficient$ \\
%\hline
%$k$ & $the \,\ permeability \,\ tensor$ \\
%\hline 
%$NC$ & 代表 $cell$ 的个数 \\
%\hline
%$NE$ & 代表总的 $edge$ 的个数 \\
%\hline
%\end{tabular}

\section{欧拉—拉格朗日方程}
欧拉—拉格朗日方程是泛函极值问题的必要条件,假设$J[y(x)]$的极值问题的解为$y=y(x)$,现在推导这个解所满足的微分方程。

$\delta J=\int_{a}^{b}(\frac{\partial f}{\partial y}\delta y + \frac{\partial f}{\partial y'}\delta y')\mathrm{d}x=0$,将第二项分部积分得到$\int_{a}^{b}(\frac{\partial f}{\partial y'}\delta y')\mathrm{d}x=\int_{a}^{b}\frac{\partial f}{\partial y'}\mathrm{d}\delta y$,因为$\delta y(a)=0$,$\delta y(b)=0$,所以$\int_{a}^{b}(\frac{\partial f}{\partial y'}\delta y')\mathrm{d}x=-\int_{a}^{b}\delta y\mathrm{d}\frac{\partial f}{\partial y'}$,因此$\delta J=\int_{a}^{b}\frac{\partial f}{\partial y}\delta y\mathrm{d}x-\int_{a}^{b}\delta y\mathrm{d}\frac{\partial f}{\partial y'}=\int_{a}^{b}(\frac{\partial f}{\partial y}-\frac{\mathrm{d}\frac{\partial f}{\partial y'}}{\mathrm{d}x})\delta y\mathrm{d}x=0$,因为对于任何函数$\delta y$都成立,故$\frac{\partial f}{\partial y}-\frac{\mathrm{d}\frac{\partial f}{\partial y'}}{\mathrm{d}x}=0$,这就是欧拉—拉格朗日方程。
%\begin{equation*}
%\begin{cases}
%\begin{aligned}
%\frac{\mu}{k}\mathbf{u} + \nabla p & = 0 \quad in \,\ \Omega = (0,1)\times (0,1) \\
%\nabla \cdot \mathbf{u} & = f \quad in \,\ \Omega \\
%\mathbf {u} & = 0 \quad on \,\ \partial \Omega
%\end{aligned}
%\end{cases}
%\end{equation*}

%且有 \\
%\begin{equation*}
%\int_{\Omega}f dxdy = 0
%\end{equation*}

%记 $u$ 为 $\mathbf{u}$ 在 $x$ 方向的分量,$v$ 为 $\mathbf{u}$ 在 $y$ 方向的分量,则有 \\

%\begin{equation*}
%\begin{cases}
%\begin{aligned}
%\frac{\mu}{k}\cdot u + \partial_x p & = 0 \quad (1) \\
%\frac{\mu}{k}\cdot v + \partial_y p & = 0 \quad (2) \\
%\partial_x u + \partial_y v & = f \quad (3)
%\end{aligned}
%\end{cases}
%\end{equation*}

%\section{离散后组装矩阵}
%利用一阶向前差分把方程变成差分方程,现在从 $edge$ 和 $cell$ 的角度考虑模型。 \\

%对于 $(1)$, 从内部纵向 $edge$ 的角度考虑:
%我们需要找到内部纵向 $edge$ 所对应的左手边的 $cell$ 和右手边的 $cell$. 左右两边的$cell$ 所对应的 $p$ 分别记为 $p_{l}$、$p_{r}$.$u$ 为 $edge$ 的中点,记为 $u_m$。按照 $mesh$ 里的编号规则排序。\\

%则每条内部边上所对应的差分方程为:

%\begin{equation*}
%\frac{\mu}{k} \cdot u_m + \frac{p_r - p_l}{hx} = 0
%\end{equation*}

%对于 $(2)$,从内部横向 $edge$ 的角度考虑:
%我们需要找到内部横向 $edge$ 所对应的左手边的 $cell$ 和右手边的 $cell$. $cell$ 所对应的 $p$ 与 $(1)$ 中的相同。$v$ 为 $edge$ 的中点,记为 $v_m$。\\

%则每条内部边上所对应的差分方程为:\\

%\begin{equation*}
%\frac{\mu}{k} \cdot v_m + \frac{p_l - p_r}{hy} = 0
%\end{equation*}

%对于 $(3)$, 从 $cell$ 的角度考虑:
%由于单元是四边形单元,我们记单元所对应边的局部编号为[0,1,2,3](StructureQuadMesh.py 里的网格),第 $i$ 个单元所对应的边记为 $e_{i,0},e_{i,1},e_{i,2},e_{i,3}$。\\

%则 $(3)$ 式第 $i$ 个单元所对应的差分方程为:\\

%\begin{equation*}
%\frac{u_{e_{i,1}} - u_{e_{i,3}}}{hx} + \frac{v_{e_{i,2}} - v_{e_{i,0}}}{hy} = f_i
%\end{equation*}

%我们需要生成一个 $(NE+NC)\times(NE+NC)$的系数矩阵,把它看成分块矩阵
%\begin{equation*}
%\begin{pmatrix}
% A_{1,1} & A_{1,2} \\
%A_{2,1} & A_{2,2}
%\end{pmatrix}
%\end{equation*}

%其中 \\

%\begin{equation*}
%\begin{aligned}
%A_{1,1} : NE\times NE \\
%A_{1,2} : NE\times NC \\
%A_{2,1} : NC\times NE \\
%A_{2,2} : NC\times NC
%\end{aligned}
%\end{equation*}

%$A_{1,1}$ 对应的是 $(1),(2)$ 两式的第一项,即含有 $u,v$ 的项,$A_12$ 对应的是 $(1),(2)$ 两式的第二项。

%\newpage
%\nocite{*}
%\bibliography{ref}
\end{document}

