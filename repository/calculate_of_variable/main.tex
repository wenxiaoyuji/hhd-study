% !Mode:: "TeX:UTF-8"
\documentclass{article}
\input{en_preamble.tex}
\input{xecjk_preamble.tex}
\setCJKmainfont{STKaiti} % 如果请替换为本地系统有的字体
%中文断行
\XeTeXlinebreaklocale "zh"
\XeTeXlinebreakskip = 0pt plus 1pt minus 0.1pt
\begin{document}
\title{用 $AMIB$ 方法求解椭圆界面问题}
%\author{李奥}
\date{\today}
\maketitle
\tableofcontents
%\newpage
\section{摘要}
本文介绍了使用分片常系数求解二维椭圆界面问题,并且这个问题有二阶精度。这种增广的 $MIB$ 与标准的 $MIB$,$IIM$, 明显跳跃$IIM$ 在一些关键性质有着无缝链接,并且有新的快速界面算法。基于 $MIB$,零阶和一阶跳跃点都会强制通过任意的凸界面,并在靠近界面的 Cartesian nodes处得到虚拟值。通过使用这样的虚拟值,提出了一个简单的程序来重建笛卡尔导数跳量作为辅助变量,并将它们与跳量校正的泰勒级数展开耦合,这使我们能够使跨越界面的中心差分是二阶的。此外,通过使用Schur补来‘解除’辅助变量和函数值的代数计算,通过使用快速傅立叶变换(FFT)可以有效地反演离散拉普拉斯算子。在我们的数值实验中发现,求解辅助系统的迭代次数与网格尺寸的关系不大。
$AMIB$ 优点\\
1).$CPU$ 的时间明显减少;
2).在处理复杂界面问题时保二阶精度; 

\section{简介}
\subsection{模型}

\begin{equation}
-\nabla \cdot (\beta\nabla u) = f(x,y),\,\ (x,y)\in \Omega
\end{equation}
Dilichlet 边界条件

\begin{equation}
u(x,y) = g(x,y),\,\ (x,y)\in \partial \Omega
\end{equation}
\subsection{符号}
\begin{tabular}{ |l|l| }   
\hline   
\multicolumn{2}{|c|}{符号说明} \\   
\hline
符号 & 含义\\
\hline
$\Omega$ & 二维长方形区域 \\
\hline
$nx$ & $x$ 方向剖分的段数 \\
\hline
$ny$ & $y$ 方向剖分的段数 \\
\hline
$hx$ &  $x$ 方向每段的长度\\
\hline
$hy$ &  $y$ 方向每段的长度 \\
\hline
$\mu$ & $the \,\ viscosity \,\ coefficient$ \\
\hline
$k$ & $the \,\ permeability \,\ tensor$ \\
\hline 
$NC$ & 代表 $cell$ 的个数 \\
\hline
$NE$ & 代表总的 $edge$ 的个数 \\
\hline
\end{tabular}

\section{模型}
\begin{equation*}
\begin{cases}
\begin{aligned}
\frac{\mu}{k}\mathbf{u} + \nabla p & = 0 \quad in \,\ \Omega = (0,1)\times (0,1) \\
\nabla \cdot \mathbf{u} & = f \quad in \,\ \Omega \\
\mathbf {u} & = 0 \quad on \,\ \partial \Omega
\end{aligned}
\end{cases}
\end{equation*}

且有 \\
\begin{equation*}
\int_{\Omega}f dxdy = 0
\end{equation*}

记 $u$ 为 $\mathbf{u}$ 在 $x$ 方向的分量,$v$ 为 $\mathbf{u}$ 在 $y$ 方向的分量,则有 \\

\begin{equation*}
\begin{cases}
\begin{aligned}
\frac{\mu}{k}\cdot u + \partial_x p & = 0 \quad (1) \\
\frac{\mu}{k}\cdot v + \partial_y p & = 0 \quad (2) \\
\partial_x u + \partial_y v & = f \quad (3)
\end{aligned}
\end{cases}
\end{equation*}

\section{离散后组装矩阵}
利用一阶向前差分把方程变成差分方程,现在从 $edge$ 和 $cell$ 的角度考虑模型。 \\

对于 $(1)$, 从内部纵向 $edge$ 的角度考虑:
我们需要找到内部纵向 $edge$ 所对应的左手边的 $cell$ 和右手边的 $cell$. 左右两边的$cell$ 所对应的 $p$ 分别记为 $p_{l}$、$p_{r}$.$u$ 为 $edge$ 的中点,记为 $u_m$。按照 $mesh$ 里的编号规则排序。\\

则每条内部边上所对应的差分方程为:

\begin{equation*}
\frac{\mu}{k} \cdot u_m + \frac{p_r - p_l}{hx} = 0
\end{equation*}

对于 $(2)$,从内部横向 $edge$ 的角度考虑:
我们需要找到内部横向 $edge$ 所对应的左手边的 $cell$ 和右手边的 $cell$. $cell$ 所对应的 $p$ 与 $(1)$ 中的相同。$v$ 为 $edge$ 的中点,记为 $v_m$。\\

则每条内部边上所对应的差分方程为:\\

\begin{equation*}
\frac{\mu}{k} \cdot v_m + \frac{p_l - p_r}{hy} = 0
\end{equation*}

对于 $(3)$, 从 $cell$ 的角度考虑:
由于单元是四边形单元,我们记单元所对应边的局部编号为[0,1,2,3](StructureQuadMesh.py 里的网格),第 $i$ 个单元所对应的边记为 $e_{i,0},e_{i,1},e_{i,2},e_{i,3}$。\\

则 $(3)$ 式第 $i$ 个单元所对应的差分方程为:\\

\begin{equation*}
\frac{u_{e_{i,1}} - u_{e_{i,3}}}{hx} + \frac{v_{e_{i,2}} - v_{e_{i,0}}}{hy} = f_i
\end{equation*}

我们需要生成一个 $(NE+NC)\times(NE+NC)$的系数矩阵,把它看成分块矩阵
\begin{equation*}
\begin{pmatrix}
A_{1,1} & A_{1,2} \\
A_{2,1} & A_{2,2}
\end{pmatrix}
\end{equation*}

其中 \\

\begin{equation*}
\begin{aligned}
A_{1,1} : NE\times NE \\
A_{1,2} : NE\times NC \\
A_{2,1} : NC\times NE \\
A_{2,2} : NC\times NC
\end{aligned}
\end{equation*}

$A_{1,1}$ 对应的是 $(1),(2)$ 两式的第一项,即含有 $u,v$ 的项,$A_12$ 对应的是 $(1),(2)$ 两式的第二项。

\newpage
\nocite{*}
\bibliography{ref}
\end{document}

