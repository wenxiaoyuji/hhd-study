% !Mode:: "TeX:UTF-8"
\documentclass[12pt,a4paper]{article}
\input{../en_preamble.tex}
\input{../xecjk_preamble.tex}

\title{有限差分法}
%\author{}
\date{\chntoday}

\begin{document}
%得到相应的线性方程组
%$$
%\textbf{Au}=\textbf{f}
%$$
%矩阵 $A$ 仍然是对称的,但只有半定的。$\textbf{A}$ 的核: $\textbf{Au}=0$ 当且仅当 $\textbf{u}=c$.这需要兼容条件($5$)的离散格式:
%\begin{equation}
%\sum_{i=1}^N f_i=0
%\end{equation}

\section{误差估计}
为了分析误差,我们需要把这个问题变成一个正常的空间。有限维线性空间 $V_h$ 的范数是最大范数:对于$v\in V_h$,
$$
\left \| v\right \|_{\infty,\Omega_h}=\max_{1\le i\le n+1,1\le j\le m+1}\lbrace |v_{i,j}|\rbrace
$$
下标 $h$ 表示这个范数依赖于三角剖分,因为对于不同的 $h$,我们有不同的 $v_{i,j}$.注意,这是 $R^N$ 的 $l^\infty$范数。

\textbf{定理3.1(离散最大值原理)}
令 $v\in V_h$ 满足
$$
\Delta _h v\ge0
$$
那么
$$
\max_{\Omega_h}v\le\max_{\Gamma_h}v
$$
等式成立当且仅当 $v$ 是常数。

证明:假设 $\max_{\Omega_h}v >\max_{\Gamma_h}v$,那么我们可以取一个内部节点 $x_0$,并且在该点达到最大值。设$x_1,x_2,x_3$ 和 $x_4$ 是 $x_0$ 的四个周围点。那么
$$
4v(x_0)=\sum_{i=1}^4 v(x_i)-h^2\Delta _h v(x_0)\le\sum_{i=1}^4 v(x_i)\le 4v(x_0)
$$
等式始终成立,并且由 $4v(x_0)\le\sum_{i=1}^4 v(x_i)$ 可知,$v(x_0)=v(x_i),i=1,2,3,4$,如果一点的值小于 $v(x_0)$,则必有另一点的值大于 $v(x_0)$,与 $x_0$ 处取得最大值矛盾,因此 $v$ 在 $x_0$ 的临近点处也取得最大值。再通过对 $x_0$ 的临近点取临近点,运用上面的方法,最终可以断定 $v$是常数,这与假设 $\max_{\Omega_h}v >\max_{\Gamma_h}v$ 矛盾。

\textbf{定理3.2}
令 $u_h$ 是下面方程的解
\begin{equation}
-\Delta_h u_h=f_I ~ at ~\Omega_h\setminus\Gamma_h,~ u_h=g_I ~ at ~ \Gamma_h
\end{equation}
那么
\begin{equation}
\left \| u_h\right \|_{\infty,\Omega_h}\le\frac{1}{8}\left \| f_I\right \|_{\infty,\Omega_h\setminus\Gamma_h}+\left \| g_I\right \|_{\Gamma_h,\infty}
\end{equation}

证明:我们引入比较函数,
$$
\phi=\frac{1}{4}\left[(x-\frac{1}{2})^2+(y-\frac{1}{2})^2\right]
$$
满足 $\Delta_h\phi_I=1$ 在 $\Omega_h\setminus\Gamma_h$,并且 $0\le\phi\le 1/8$.令 $M=\parallel f_I\parallel _{\infty,\Omega_h\setminus\Gamma_h}$,那么
$$
\Delta_h(u_h+M\phi_I)=\Delta_h u_h+M\ge 0
$$
所以,
$$
\max_{\Omega_h}u_h\le\max_{\Omega_h}(u_h+M\phi_I)\le\max_{\Gamma_h}(u_h+M\phi_I)\le\max_{\Gamma_h}g_I+\frac{1}{8}M
$$
所以,$u_h$ 由($2$)限定。

\textbf{推论3.3}
设 $u$ 是 $Dirichlet$ 问题的真解,$u_h$ 是离散问题($1$)的解。那么
$$
\left \| u_I-u_h\right \|_{\infty,\Omega_h}\le\frac{1}{8}\left \|\Delta_h u_I-(\Delta u)_I\right \|_{\infty,\Omega_h\setminus \Gamma_h}
$$
下面研究误差 $\left \|\Delta_h u_I-(\Delta u)_I\right \|_{h,\infty}$ .

\textbf{引理3.4}
如果 $u\in C^4(\Omega)$,那么
$$
\left \|\Delta_h u_I-(\Delta u)_I\right \|_{\infty,\Omega_h \setminus \Gamma_h}\le\frac{h^2}{6}max\left \{\left \|\frac{\partial ^4u}{\partial x^4}\right \|_{\infty,\Omega},\left \|\frac{\partial ^4u}{\partial y^4}\right \|_{\infty}\right \}
$$
证明:
$$
u(x_{i+1},y_j)=u(x_i,y_j)+h_x\frac{\partial u(x_i,y_j)}{\partial x}+\frac{h_x^2}{2!}\frac{\partial ^2 u(x_i,y_j)}{\partial x^2}+\frac{h_x^3}{3!}\frac{\partial ^3 u(x_i,y_j)}{\partial x^3}+\frac{h_x^4}{4!}\frac{\partial ^4 u(x_i,y_j)}{\partial x^4}
$$

$$
u(x_{i-1},y_j)=u(x_i,y_j)-h_x\frac{\partial u(x_i,y_j)}{\partial x}+\frac{h_x^2}{2!}\frac{\partial ^2 u(x_i,y_j)}{\partial x^2}-\frac{h_x^3}{3!}\frac{\partial ^3 u(x_i,y_j)}{\partial x^3}+\frac{h_x^4}{4!}\frac{\partial ^4 u(x_i,y_j)}{\partial x^4}
$$

因此
$$
\frac{u(x_{i+1},y_j)-2u(x_i,y_j)+u(x_{i-1},y_j)}{h_x^2}=\frac{\partial ^2 u(x_i,y_j)}{\partial x^2}+\frac{h_x^2}{12}\frac{\partial ^4 u(x_i,y_j)}{\partial x^4}
$$
同理可得
$$
\frac{u(x_i,y_{j+1})-2u(x_i,y_j)+u(x_i,y_{j-1})}{h_x^2}=\frac{\partial ^2 u(x_i,y_j)}{\partial y^2}+\frac{h_y^2}{12}\frac{\partial ^4 u(x_i,y_j)}{\partial y^4}
$$
因此有
$$\frac{h_x^2}{12}\frac{\partial ^4 u(x_i,y_j)}{\partial x^4}+\frac{h_y^2}{12}\frac{\partial ^4 u(x_i,y_j)}{\partial y^4}\le\frac{h^2}{12}(\frac{\partial ^4 u(x_i,y_j)}{\partial x^4}+\frac{\partial ^4 u(x_i,y_j)}{\partial y^4})
$$





\textbf{定理3.5}
设 $u$ 是 $Dirichlet$ 问题的真解,以及离散问题($1$)的解。如果 $u\in C^4(\Omega)$,那么
$$
\left \| u_I-u_h\right \|_{\infty,\Omega_h}\le Ch^2
$$
常数 $C$
$$
C=\frac{1}{48}max\left \{\left \|\frac{\partial ^4u}{\partial x^4}\right \|_{\infty,\Omega},\left \|\frac{\partial ^4u}{\partial y^4}\right \|_{\infty}\right \}
$$
\textbf{定理3.5}可由\textbf{推论3.3}和\textbf{引理3.4}得到。

\section{单元中心有限差分法}
在一些实际应用中,比如计算流体动力学 $(CFD)$,泊松方程是在稍微不同的网格上求解的(以单元为中心的均匀网格),如下图。在这一节中,我们
考虑单元处泊松方程的有限差分。

\begin{figure}[H]
\centering
\includegraphics[scale=0.5]{./figures/1.png}
\caption{}
\end{figure}

在内部节点,可以使用标准模板 $(4,−1,−1,−1,−1)$,但靠近边界的节点不能采用上面的模板。内节点之间
的轴向距离仍为 $h$,但靠近边界的节点(单元接触边界的中心)与边界之间的距离为 $h/2$.可以证明,对于 $Neumann$ 边界条件,靠近边界节点的模板是 $(3,−1,−1,−1)$ 和角单元节点是 $(2,−1,−1)$.

处理单元中心差分的 $Dirichlet$ 边界条件时,我们仍然可以引入鬼网格点,并使用标准的 $(4,−1,−1,−1)$ 模
板用于靠近边界的节点。边界上并没有网格点,鬼点处的值需要消除,即需要 $(u_{0,j}+u_{1,j})/2=g_{0,j}:=g_{1/2,j}$.
$$
\frac{5u_{1,j}-u_{2,j}-u_{1,j-1}-u_{1,j+1}}{h^2}=f_{1,j}+\frac{2g_{1/2,j}}{h^2}
$$
对于靠近边界节点,模板是 $(5,-1,-1,-1,-2)$;对于角单元节点,模板是 $(6,-1,-1,-2,-2)$.

但是这种处理是低阶的,为了获得更好的截断误差,可以采用下面的方法,用 $u_{1/2,j}$,$u_{1,j}$,$u_{2,j}$ 处理 $u_{0,j}$.即用 $u_{1/2,j}$,$u_{1,j}$,$u_{2,j}$ 构造一个二次拉格朗日插值多项式,令 $X=X(x,y)$ 表示点的坐标,因此有

$$
l_{1/2,j}(X)=\frac{(X-X_{1,j})(X-X_{2,j})}{(X_{1/2,j}-X_{1,j})(X_{1/2,j}-X_{2,j})}
$$

$$
l_{1,j}(X)=\frac{(X-X_{1/2,j})(X-X_{2,j})}{(X_{1,j}-X_{1/2,j})(X_{1,j}-X_{2,j})}
$$

$$
l_{2,j}(X)=\frac{(X-X_{1/2,j})(X-X_{1,j})}{(X_{2,j}-X_{1/2,j})(X_{2,j}-X_{1,j})}
$$


$$
L_2(X)=u_{1/2,j} l_{1/2,j}(X)+u_{1,j} l_{1,j}(X)+u_{2,j} l_{2,j}(X)
$$

把 $u_{0,j}$ 的坐标带入上式,得到
$$
u_{0,j}=-2u_{1,j}+\frac{1}{3}u_{2,j}+\frac{8}{3}u_{1/2,j}
$$
修改后的方案
$$
\frac{ 6u_{1,j}-\frac{4}{3}u_{2,j}-u_{1,j-1}-u_{1,j+1} }{h^2}=f_{1,j}+\frac{ \frac{8}{3} g_{1/2,j} }{h^2}
$$

我们用 $(6,−\frac{4}{3},−1,−1,−\frac{8}{3})$ 表示靠边界节点的模板。由于截断误差的改善,上面的方法可以提高收敛速度,但是破坏了矩阵的对称性。

对于泊松方程,有一种方法可以保持截断误差和对称性。为简单起见,我们考虑齐次 $Dirichlet$边界条件,即 $u|_{\partial\Omega}=0$,沿边界的切向导数 $\partial ^2_t u=0$.假设方程 $-\Delta u= f$ 也在边界上成立。 注意在边界上,
$\Delta$ 运算符可以写成 $\partial ^2_t+\partial ^2_n$,然后我们在 $\partial\Omega$ 上得到 $\partial ^2_n u=\pm f$.

然后,我们可以用 $u_1,u_{1/2}=0$ 和 $\partial ^2_n=f$ 该函数,并得到鬼点处的一个方程
$$
u_{1,j}+u_{0,j}=\frac{h^2}{4}f_{1/2,j}
$$
并将边界模板修改为
$$
\frac{5u_{1,j}-u_{2,j}-u_{1,j-1}-u_{1,j+1}}{h^2}=f_{1,j}+\frac{1}{4}f_{1/2,j}
$$













%\input{test.tex}

%\cite{tam19912d}
%\bibliography{../ref}
\end{document}
