%%%%%%%%------------------------------------------------------------------------
%%%% xeCJK相关宏包

\usepackage{xltxtra,fontspec,xunicode}
\usepackage[slantfont, boldfont]{xeCJK} 

\setlength{\parindent}{2em}%中文缩进两个汉字位

%% 针对中文进行断行
\XeTeXlinebreaklocale "zh"             

%% 给予TeX断行一定自由度
\XeTeXlinebreakskip = 0pt plus 1pt minus 0.1pt

%%%% xeCJK设置结束                                       
%%%%%%%%------------------------------------------------------------------------

%%%%%%%%------------------------------------------------------------------------
%%%% xeCJK字体设置

%% 设置中文标点样式,支持quanjiao、banjiao、kaiming等多种方式
\punctstyle{kaiming}                                        
                                                     
%% 设置缺省中文字体
%\setCJKmainfont[BoldFont={Adobe Heiti Std}, ItalicFont={Adobe Kaiti Std}]{Adobe Song Std}   
\setCJKmainfont{SimSun}
%% 设置中文无衬线字体
%\setCJKsansfont[BoldFont={Adobe Heiti Std}]{Adobe Kaiti Std}  
%% 设置等宽字体
%\setCJKmonofont{Adobe Heiti Std}                            

%% 英文衬线字体
\setmainfont{DejaVu Serif}                                  
%% 英文等宽字体
\setmonofont{DejaVu Sans Mono}                              
%% 英文无衬线字体
\setsansfont{DejaVu Sans}                                   

%% 定义新字体
\setCJKfamilyfont{song}{Adobe Song Std}                     
\setCJKfamilyfont{kai}{Adobe Kaiti Std}
\setCJKfamilyfont{hei}{Adobe Heiti Std}
\setCJKfamilyfont{fangsong}{Adobe Fangsong Std}
\setCJKfamilyfont{lisu}{LiSu}
\setCJKfamilyfont{youyuan}{YouYuan}

%% 自定义宋体
\newcommand{\song}{\CJKfamily{song}}                       
%% 自定义楷体
\newcommand{\kai}{\CJKfamily{kai}}                         
%% 自定义黑体
\newcommand{\hei}{\CJKfamily{hei}}                         
%% 自定义仿宋体
\newcommand{\fangsong}{\CJKfamily{fangsong}}               
%% 自定义隶书
\newcommand{\lisu}{\CJKfamily{lisu}}                       
%% 自定义幼圆
\newcommand{\youyuan}{\CJKfamily{youyuan}}                 

%%%% xeCJK字体设置结束
%%%%%%%%------------------------------------------------------------------------

%%%%%%%%------------------------------------------------------------------------
%%%% 一些关于中文文档的重定义
\newcommand{\chntoday}{\number\year\,年\,\number\month\,月\,\number\day\,日}
%% 数学公式定理的重定义

%% 中文破折号,据说来自清华模板
\newcommand{\pozhehao}{\kern0.3ex\rule[0.8ex]{2em}{0.1ex}\kern0.3ex}

\newtheorem{example}{例}                                   
\newtheorem{theorem}{定理}[section]                         
\newtheorem{definition}{定义}
\newtheorem{axiom}{公理}
\newtheorem{property}{性质}
\newtheorem{proposition}{命题}
\newtheorem{lemma}{引理}
\newtheorem{corollary}{推论}
\newtheorem{remark}{注解}
\newtheorem{condition}{条件}
\newtheorem{conclusion}{结论}
\newtheorem{assumption}{假设}

\makeatletter %
\@ifclassloaded{beamer}{

}{
%% 章节等名称重定义
\renewcommand{\contentsname}{目录}     
\renewcommand{\indexname}{索引}
\renewcommand{\listfigurename}{插图目录}
\renewcommand{\listtablename}{表格目录}
\renewcommand{\appendixname}{附录}
\renewcommand{\appendixpagename}{附录}
\renewcommand{\appendixtocname}{附录}
%% 设置chapter、section与subsection的格式
\titleformat{\chapter}{\centering\huge}{第\thechapter{}章}{1em}{\textbf}
\titleformat{\section}{\centering\sihao}{\thesection}{1em}{\textbf}
\titleformat{\subsection}{\xiaosi}{\thesubsection}{1em}{\textbf}
\titleformat{\subsubsection}{\xiaosi}{\thesubsubsection}{1em}{\textbf}

\@ifclassloaded{book}{

}{
\renewcommand{\abstractname}{摘要}
}
}
\makeatother

\renewcommand{\figurename}{图}
\renewcommand{\tablename}{表}

\makeatletter
\@ifclassloaded{book}{
\renewcommand{\bibname}{参考文献}
}{
\renewcommand{\refname}{参考文献} 
}
\makeatother

\floatname{algorithm}{算法}
\renewcommand{\algorithmicrequire}{\textbf{输入:}}
\renewcommand{\algorithmicensure}{\textbf{输出:}}

%%%% 中文重定义结束
%%%%%%%%------------------------------------------------------------------------
