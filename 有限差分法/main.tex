% !Mode:: "TeX:UTF-8"
\documentclass[12pt,a4paper]{article}

%%%%%%%%------------------------------------------------------------------------
%%%% 日常所用宏包

%% 控制页边距
% 如果是beamer文档类, 则不用geometry
\makeatletter
\@ifclassloaded{beamer}{}{\usepackage[top=2.5cm, bottom=2.5cm, left=2.5cm, right=2.5cm]{geometry}}
\makeatother

%% 控制项目列表
\usepackage{enumerate}

%% 多栏显示
\usepackage{multicol}

%% 算法环境
\usepackage{algorithm}  
\usepackage{algorithmic} 
\usepackage{float} 

%% 网址引用
\usepackage{url}

%% 控制矩阵行距
\renewcommand\arraystretch{1.4}

%% hyperref宏包,生成可定位点击的超链接,并且会生成pdf书签
\makeatletter
\@ifclassloaded{beamer}{
\usepackage{hyperref}
\usepackage{ragged2e} % 对齐
}{
\usepackage[%
    pdfstartview=FitH,%
    CJKbookmarks=true,%
    bookmarks=true,%
    bookmarksnumbered=true,%
    bookmarksopen=true,%
    colorlinks=true,%
    citecolor=blue,%
    linkcolor=blue,%
    anchorcolor=green,%
    urlcolor=blue%
]{hyperref}
}
\makeatother



\makeatletter % 如果是 beamer 不需要下面两个包
\@ifclassloaded{beamer}{
\mode<presentation>
{
} 
}{
%% 控制标题
\usepackage{titlesec}
%% 控制目录
\usepackage{titletoc}
}
\makeatother

%% 控制表格样式
\usepackage{booktabs}

%% 控制字体大小
\usepackage{type1cm}

%% 首行缩进,用\noindent取消某段缩进
\usepackage{indentfirst}

%% 支持彩色文本、底色、文本框等
\usepackage{color,xcolor}

%% AMS LaTeX宏包: http://zzg34b.w3.c361.com/package/maths.htm#amssymb
\usepackage{amsmath,amssymb}
%% 多个图形并排
\usepackage{subfig}
%%%% 基本插图方法
%% 图形宏包
\usepackage{graphicx}
\newcommand{\red}[1]{\textcolor{red}{#1}}
\newcommand{\blue}[1]{\structure{#1}}
\newcommand{\brown}[1]{\textcolor{brown}{#1}}
\newcommand{\green}[1]{\textcolor{green}{#1}}


%%%% 基本插图方法结束

%%%% pgf/tikz绘图宏包设置
\usepackage{pgf,tikz}
\usetikzlibrary{shapes,automata,snakes,backgrounds,arrows}
\usetikzlibrary{mindmap}
%% 可以直接在latex文档中使用graphviz/dot语言,
%% 也可以用dot2tex工具将dot文件转换成tex文件再include进来
%% \usepackage[shell,pgf,outputdir={docgraphs/}]{dot2texi}
%%%% pgf/tikz设置结束


\makeatletter % 如果是 beamer 不需要下面两个包
\@ifclassloaded{beamer}{

}{
%%%% fancyhdr设置页眉页脚
%% 页眉页脚宏包
\usepackage{fancyhdr}
%% 页眉页脚风格
\pagestyle{plain}
}

%% 有时会出现\headheight too small的warning
\setlength{\headheight}{15pt}

%% 清空当前页眉页脚的默认设置
%\fancyhf{}
%%%% fancyhdr设置结束


\makeatletter % 对 beamer 要重新设置
\@ifclassloaded{beamer}{

}{
%%%% 设置listings宏包用来粘贴源代码
%% 方便粘贴源代码,部分代码高亮功能
\usepackage{listings}

%% 设置listings宏包的一些全局样式
%% 参考http://hi.baidu.com/shawpinlee/blog/item/9ec431cbae28e41cbe09e6e4.html
\lstset{
showstringspaces=false,              %% 设定是否显示代码之间的空格符号
numbers=left,                        %% 在左边显示行号
numberstyle=\tiny,                   %% 设定行号字体的大小
basicstyle=\footnotesize,                    %% 设定字体大小\tiny, \small, \Large等等
keywordstyle=\color{blue!70}, commentstyle=\color{red!50!green!50!blue!50},
                                     %% 关键字高亮
frame=shadowbox,                     %% 给代码加框
rulesepcolor=\color{red!20!green!20!blue!20},
escapechar=`,                        %% 中文逃逸字符,用于中英混排
xleftmargin=2em,xrightmargin=2em, aboveskip=1em,
breaklines,                          %% 这条命令可以让LaTeX自动将长的代码行换行排版
extendedchars=false                  %% 这一条命令可以解决代码跨页时,章节标题,页眉等汉字不显示的问题
}}
\makeatother
%%%% listings宏包设置结束


%%%% 附录设置
\makeatletter % 对 beamer 要重新设置
\@ifclassloaded{beamer}{

}{
\usepackage[title,titletoc,header]{appendix}
}
\makeatother
%%%% 附录设置结束


%%%% 日常宏包设置结束
%%%%%%%%------------------------------------------------------------------------


%%%%%%%%------------------------------------------------------------------------
%%%% 英文字体设置结束
%% 这里可以加入自己的英文字体设置
%%%%%%%%------------------------------------------------------------------------

%%%%%%%%------------------------------------------------------------------------
%%%% 设置常用字体字号,与MS Word相对应

%% 一号, 1.4倍行距
\newcommand{\yihao}{\fontsize{26pt}{36pt}\selectfont}
%% 二号, 1.25倍行距
\newcommand{\erhao}{\fontsize{22pt}{28pt}\selectfont}
%% 小二, 单倍行距
\newcommand{\xiaoer}{\fontsize{18pt}{18pt}\selectfont}
%% 三号, 1.5倍行距
\newcommand{\sanhao}{\fontsize{16pt}{24pt}\selectfont}
%% 小三, 1.5倍行距
\newcommand{\xiaosan}{\fontsize{15pt}{22pt}\selectfont}
%% 四号, 1.5倍行距
\newcommand{\sihao}{\fontsize{14pt}{21pt}\selectfont}
%% 半四, 1.5倍行距
\newcommand{\bansi}{\fontsize{13pt}{19.5pt}\selectfont}
%% 小四, 1.5倍行距
\newcommand{\xiaosi}{\fontsize{12pt}{18pt}\selectfont}
%% 大五, 单倍行距
\newcommand{\dawu}{\fontsize{11pt}{11pt}\selectfont}
%% 五号, 单倍行距
\newcommand{\wuhao}{\fontsize{10.5pt}{10.5pt}\selectfont}
%%%%%%%%------------------------------------------------------------------------


%% 设定段间距
\setlength{\parskip}{0.5\baselineskip}

%% 设定行距
\linespread{1}


%% 设定正文字体大小
% \renewcommand{\normalsize}{\sihao}

%制作水印
\RequirePackage{draftcopy}
\draftcopyName{XTUMESH}{100}
\draftcopySetGrey{0.90}
\draftcopyPageTransform{40 rotate}
\draftcopyPageX{350}
\draftcopyPageY{80}

%%%% 个性设置结束
%%%%%%%%------------------------------------------------------------------------


%%%%%%%%------------------------------------------------------------------------
%%%% bibtex设置

%% 设定参考文献显示风格
% 下面是几种常见的样式
% * plain: 按字母的顺序排列,比较次序为作者、年度和标题
% * unsrt: 样式同plain,只是按照引用的先后排序
% * alpha: 用作者名首字母+年份后两位作标号,以字母顺序排序
% * abbrv: 类似plain,将月份全拼改为缩写,更显紧凑
% * apalike: 美国心理学学会期刊样式, 引用样式 [Tailper and Zang, 2006]

\makeatletter
\@ifclassloaded{beamer}{
\bibliographystyle{apalike}
}{
\bibliographystyle{unsrt}
}
\makeatother


%%%% bibtex设置结束
%%%%%%%%------------------------------------------------------------------------

%%%%%%%%------------------------------------------------------------------------
%%%% xeCJK相关宏包

\usepackage{xltxtra,fontspec,xunicode}
\usepackage[slantfont, boldfont]{xeCJK} 

%% 针对中文进行断行
\XeTeXlinebreaklocale "zh"             

%% 给予TeX断行一定自由度
\XeTeXlinebreakskip = 0pt plus 1pt minus 0.1pt

%%%% xeCJK设置结束                                       
%%%%%%%%------------------------------------------------------------------------

%%%%%%%%------------------------------------------------------------------------
%%%% xeCJK字体设置

%% 设置中文标点样式,支持quanjiao、banjiao、kaiming等多种方式
\punctstyle{kaiming}                                        
                                                     
%% 设置缺省中文字体
\setCJKmainfont[BoldFont={Adobe Heiti Std}, ItalicFont={Adobe Kaiti Std}]{Adobe Song Std}   
%% 设置中文无衬线字体
\setCJKsansfont[BoldFont={Adobe Heiti Std}]{Adobe Kaiti Std}  
%% 设置等宽字体
\setCJKmonofont{Adobe Heiti Std}                            

%% 英文衬线字体
\setmainfont{DejaVu Serif}                                  
%% 英文等宽字体
\setmonofont{DejaVu Sans Mono}                              
%% 英文无衬线字体
\setsansfont{DejaVu Sans}                                   

%% 定义新字体
\setCJKfamilyfont{song}{Adobe Song Std}                     
\setCJKfamilyfont{kai}{Adobe Kaiti Std}
\setCJKfamilyfont{hei}{Adobe Heiti Std}
\setCJKfamilyfont{fangsong}{Adobe Fangsong Std}
\setCJKfamilyfont{lisu}{LiSu}
\setCJKfamilyfont{youyuan}{YouYuan}

%% 自定义宋体
\newcommand{\song}{\CJKfamily{song}}                       
%% 自定义楷体
\newcommand{\kai}{\CJKfamily{kai}}                         
%% 自定义黑体
\newcommand{\hei}{\CJKfamily{hei}}                         
%% 自定义仿宋体
\newcommand{\fangsong}{\CJKfamily{fangsong}}               
%% 自定义隶书
\newcommand{\lisu}{\CJKfamily{lisu}}                       
%% 自定义幼圆
\newcommand{\youyuan}{\CJKfamily{youyuan}}                 

%%%% xeCJK字体设置结束
%%%%%%%%------------------------------------------------------------------------

%%%%%%%%------------------------------------------------------------------------
%%%% 一些关于中文文档的重定义
\newcommand{\chntoday}{\number\year\,年\,\number\month\,月\,\number\day\,日}
%% 数学公式定理的重定义

%% 中文破折号,据说来自清华模板
\newcommand{\pozhehao}{\kern0.3ex\rule[0.8ex]{2em}{0.1ex}\kern0.3ex}

\newtheorem{example}{例}                                   
\newtheorem{theorem}{定理}[section]                         
\newtheorem{definition}{定义}
\newtheorem{axiom}{公理}
\newtheorem{property}{性质}
\newtheorem{proposition}{命题}
\newtheorem{lemma}{引理}
\newtheorem{corollary}{推论}
\newtheorem{remark}{注解}
\newtheorem{condition}{条件}
\newtheorem{conclusion}{结论}
\newtheorem{assumption}{假设}

\makeatletter %
\@ifclassloaded{beamer}{

}{
%% 章节等名称重定义
\renewcommand{\contentsname}{目录}     
\renewcommand{\indexname}{索引}
\renewcommand{\listfigurename}{插图目录}
\renewcommand{\listtablename}{表格目录}
\renewcommand{\appendixname}{附录}
\renewcommand{\appendixpagename}{附录}
\renewcommand{\appendixtocname}{附录}
%% 设置chapter、section与subsection的格式
\titleformat{\chapter}{\centering\huge}{第\thechapter{}章}{1em}{\textbf}
\titleformat{\section}{\centering\sihao}{\thesection}{1em}{\textbf}
\titleformat{\subsection}{\xiaosi}{\thesubsection}{1em}{\textbf}
\titleformat{\subsubsection}{\xiaosi}{\thesubsubsection}{1em}{\textbf}

\@ifclassloaded{book}{

}{
\renewcommand{\abstractname}{摘要}
}
}
\makeatother

\renewcommand{\figurename}{图}
\renewcommand{\tablename}{表}

\makeatletter
\@ifclassloaded{book}{
\renewcommand{\bibname}{参考文献}
}{
\renewcommand{\refname}{参考文献} 
}
\makeatother

\floatname{algorithm}{算法}
\renewcommand{\algorithmicrequire}{\textbf{输入:}}
\renewcommand{\algorithmicensure}{\textbf{输出:}}

%%%% 中文重定义结束
%%%%%%%%------------------------------------------------------------------------


\title{有限差分法}
%\author{}
\date{\chntoday}

\begin{document}
\maketitle
有限差分法,用差商代替每一个导数。从某种意义上说,有限差分公式比基于其他公式的方法更能直接地解决偏微分方程的数值解问题。有限差分法的主要缺点是灵活性。

\section{有限差分法}
在本节中,为了简单起见,我们讨论了单位平方 $\Omega=(0,1)\times (0,1)$ 上的 $Poisson$ 方程。变系数和更复杂的区域将在有限元方法中讨论。此外,我们假设 $u$ 足够光滑,使我们能够自由地使用 $Taylor$ 展式。

给定两个整数 $m,n\ge 2$,我们利用 $(0,1):\lbrace x_i=(i-1)h_x,i=1,\cdots ,m,h_x=1/(m-1)\rbrace,\lbrace y_j=(j-1)h_y,j=1,\cdots ,n,h_y=1/(n-1)\rbrace$ 的两个网格的张量积构造了矩形网格 $\tau_h$,令 $h=max\lbrace h_x,h_y\rbrace$ 表示 $\tau_h$ 的大小,记 $\Omega_h=\lbrace (x_i,y_j)\in \Omega\rbrace$,边界 $\Gamma_h=\lbrace (x_i,y_j)\in \partial\Omega\rbrace$.

我们考虑与 $R^N,N=m\times n$ 同构的离散函数空间 $V_h=\lbrace u_h(x_i,y_j),1\le i\le m,1\le j\le n\rbrace$,用 $u_{i,j}$ 表示 $u_h(x_i,y_j)$.对于连续函数 $u\in C(\Omega)$,插值算子 $I_h:C(\Omega)\rightarrow V_h$ 将 $u$ 映射为离散函数,用 $u_I$ 表示,即 $(u_I)_{i,j}=u(x_i,y_j)$,注意,离散函数的值仅在网格点定义。

类似的定义也适用于一维情况。选择网格大小 $h$ 和 $u\in V_h(0,1)$,在内部节点 $x_j$ 处的离散公式包括:

向后差分:$(D^-u)_j=\frac{u_j-u_{j-1}}{h}$

向前差分:$(D^+u)_j=\frac{u_{j+1}-u_j}{h}$

中心差分:$(D^\pm u)_j=\frac{u_{j+1}-u_{j-1}}{2h}$

二阶中心差分:$(D^2 u)_j=\frac{u_{j+1}-2u_j+u_{j-1}}{h^2}$

利用 $Talyor$ 展式容易证明:
$$
(D^-u)_j-u'(x_j)=O(h),~(D^+u)_j-u'(x_j)=O (h)
$$
$$
(D^\pm u)_j-u'(x_j)=O (h^2),~(D^2 u)_j-u'(x_j)=O (h^2)
$$

我们将使用这些差分公式,特别是第二个中心差分来逼近内部节点上的 $Laplace$ 算子$(x_i,y_j)$:
$$
(\Delta _h u)_{i,j}=(D^2_{xx}u)_{i,j}+(D^2_{yy}u)_{i,j}= \frac{u_{i+1,j}-2u_{i,j}+u_{i-1,j}}{h^2_x}+\frac{u_{i,j+1}-2u_{i,j}+u_{i,j-1}}{h^2_y} 
$$
它被称为五点格式,因为只涉及五个点。当 $h_x=h_y$ 时,
\begin{equation}
-(\Delta _h u)_{i,j}=\frac{4u_{i,j}-u_{i+1,j}-u_{i-1,j}-u_{i,j+1}-u_{i,j-1}}{h^2} 
\end{equation}
对于右边的函数 $f$,我们只取节点值,即 $f_{i,j}=(f_I)_{i,j}=f(x_i,y_j)$

求解泊松方程的有限差分法
\begin{equation}
-(\Delta _h u)_{i,j}=f(x_i,y_j),1\le i\le m,1\le j\le n
\end{equation}

让我们给出 $N=m\times n$ 网格的排序,并使用单个索引 $k=1$ 到 $N$,即 $u_k=u_{i(k),j(k)},k\rightarrow (i(k),j(k))$ 

\begin{equation}
Au=f
\end{equation}
$A\in R^{N\times N},u\in R^N,f\in R^N$

\textbf{注1.1}
网格点可以存在不同的排序。

\section{边界条件}
我们将讨论如何在有限差分法中处理边界条件。$Dirichlet$ 边界条件相对简单,$Neumann$ 边界条件处理时,需要不存在的点($ghost$ 点)。

具有 $Dirichlet$ 边界条件的 $Poisson$ 方程
\begin{equation}
-\Delta u=f ~in~ \Omega ,~u=g ~on~ \Gamma =\partial\Omega
\end{equation}
边界上的值由边界条件给出。即对于 $(x_i,y_j)\in\partial\Omega$,有 $u_{i,j}=g(x_i,y_j)$.有几种方法可以施加 $Dirichlet$ 边界条件。

一种方法是令 $a_{ii}=1,a_{ij}=0,j\ne i$,当 $x_i\in\Gamma$ 时,$f_i=g(x_i)$.

具有 $Neumann$ 边界条件的 $Poisson$ 方程
$$
-\Delta u=f ~in~ \Omega ,\frac{\partial u}{\partial\overrightarrow{n}}=g ~ on ~\Omega
$$
$f$ 和 $g$ 有一个兼容的条件:
\begin{equation}
-\int_{\Omega} f\, \mathrm{d}x=\int_{\Omega} \Delta u\, \mathrm{d}x=\int_{\partial\Omega}\frac{\partial u}{\partial\overrightarrow{n}}\, \mathrm{d}S=\int_{\partial\Omega} g\, \mathrm{d}S
\end{equation}
对正常导数的自然近似利用单侧方向导数,例如:
$$
\frac{\partial u}{\partial\overrightarrow{n}}(x_1,y_j)=\frac{u_{1,j}-u_{2,j}}{h}+O(h)
$$
但这只是一阶近似。为了更准确地处理 $Neumann$ 边界条件,我们引入了区域外和边界旁边的不存在的点($ghost$ 点)。

我们允许扩展索引 $0\le i,j\le n+1$,然后我们就可以使用中心差分:
$$
\frac{\partial u}{\partial\overrightarrow{n}}(x_1,y_j)=\frac{u_{0,j}-u_{2,j}}{2h}+O(h^2)
$$
$u_{0,j}$ 没有定义,我们需要把它从等式中消除。这是可能的,因为在边界点$(x_1,y_j)$,我们有两个方程:
\begin{equation}
4u_{1,j}-u_{2,j}-u_{0,j}-u_{1,j+1}-u_{1,j-1}=h^2f_{1,j}
\end{equation}

\begin{equation}
u_{0,j}-u_{2,j}=2hg_{1,j}
\end{equation}
由($7$)式,我们得到 $u_{0,j}=2hg_{1,j}+u_{2,j}$,将其代入($6$)并乘以 $1/2$,我们在点 $(x_1,y_j)$得到方程:
$$
2u_{1,j}-u_{2,j}-0.5u_{1,j+1}-0.5u_{1,j-1}=0.5h^2f_{1,j}+hg_{1,j}
$$
为了保持矩阵的对称性。除了四个角点外,我们还可以用同样的方法处理其他边界点。在角点,范数向量也没有很好的定义。我们将利用两个方向导数的近似值。以 $(0,0)$ 为例,我们有
\begin{equation}
4u_{1,1}-u_{2,1}-u_{0,1}-u_{1,2}-u{1,0}=h^2f_{1,1}
\end{equation}

\begin{equation}
u_{0,1}-u_{2,1}=2hg_{1,1}
\end{equation}

\begin{equation}
u_{1,0}-u_{1,2}=2hg_{1,1}
\end{equation}

因此,我们可以从($9$)和($10$)中求解 $u_{0,1}$ 和 $u_{1,0}$,并将它们带入($8$).同样,为了保持矩阵的对称性,我们将($8$)乘以$1/4$,这给出了角点$(x_1,y_1)$的方程。
$$
u_{1,1}-0.5u_{2,1}-0.5u_{1,2}=0.25h^2f_{1,1}+hg_{1,1}
$$
类似的方式处理其他角点。

%得到相应的线性方程组
%$$
%\textbf{Au}=\textbf{f}
%$$
%矩阵 $A$ 仍然是对称的,但只有半定的。$\textbf{A}$ 的核: $\textbf{Au}=0$ 当且仅当 $\textbf{u}=c$.这需要兼容条件($5$)的离散格式:
%\begin{equation}
%\sum_{i=1}^N f_i=0
%\end{equation}

%\section{误差估计}
%为了分析误差,我们需要把这个问题变成一个正常的空间。有限线性空间 $V_h$ 的范数是最大范数:对于$v\in V_h$,
%$$
%\left \| v\right \|_{\infty,\Omega_h}=\max_{1\le i\le n+1,1\le j\le m+1}\lbrace |v_{i,j}|\rbrace
%$$
%下标 $h$ 表示这个范数依赖于三角剖分,因为对于不同的 $h$,我们有不同的 $v_{i,j}$.注意,这是 $R^N$ 的 $l^\infty$范数。

%可以证明 $\Delta ^{-1}_h:(V_h,\left \|\cdot\right \|_{\infty,\Omega_h})\rightarrow (V_h,\left \|\cdot\right \|_{\infty,\Omega_h})$ 对 $h$ 是一致稳定的

%\textbf{定理3.1(离散最大值原理)}
%令 $v\in V_h$ 满足
%$$
%\Delta _h v\ge0
%$
%那么
%$$
%\max_{\Omega_h}v\le\max_{\Gamma_h}v
%$$
%等式成立当且仅当 $v$ 是常数。

%证明:假设 $\max_{\Omega_h}v >\max_{\Gamma_h}v$,那么我们可以取一个内部节点 $x_0$,并且在该点达到最大值。设$x_1,x_2,x_3$ 和 $x_4$ 是 $x_0$ 的四个周围点。那么
%$$
%4v(x_0)=\sum_{i=1}{4}v(x_i)-h^2\Delta _h v(x_0)\le\sum_{i=1}{4}v(x_i)\le 4v(x_0)
%$$
%因此,在 $x_0$ 所有最临近的点处,也是 $x_0$.向内部的临近的点应用相同的参数,等等,然后我们可以断定 $v$是常数,这与假设 $\max_{\Omega_h}v >\max_{\Gamma_h}v$ 矛盾。第二个部分可以通过类似的参数证明。

%\textbf{定理3.2}
%令 $u_h$ 是下面方程的解
%\begin{equation}
%-\Delta_h u_h=f_I ~ at ~\Omega_h\setminus\Gamma_h,~ u_h=g_I ~ at ~ \Gamma_h
%\end{equation}
%那么
%\begin{equation}
%\left \| u_h\right \|_{\infty,\Omega_h}\le\frac{1}{8}\left \| f_I\right \|_{\infty,\Omega_h\setminus\Gamma_h}+\left \| g_I\right \|_{\Gamma_h,\infty}
%\end{equation}

%证明:我们引入比较函数,
%$$
%\phi=\frac{1}{4}\left[(x-\frac{1}{2})^2+(y-\frac{1}{2})^2\right]
%$$
%满足 $\Delta_h\phi_I=1$ 在 $\Omega_h\setminus\Gamma_h$,并且 $0\le\phi\le 1/8$.令 $M=\parallel f_I\parallel _{\infty,\Omega_h\setminus\Gamma_h}$,那么
%$$
%\Delta_h(u_h+M\phi_I)=\Delta_h u_h+M\ge 0
%$$
%所以,
%$$
%\max_{\Omega_h}u_h\le\max_{\Omega_h}(u_h+M\phi_I)\le\max_{\Gamma_h}(u_h+M\phi_I)\le\max_{\Gamma_h}g_I+\frac{1}{8}M
%$$
%所以,$u_h$ 由($13$)限定。

%\textbf{推论3.3}
%设 $u$ 是 $Dirichlet$ 问题($4$)的解,以及离散问题($12$)的解。那么
%$$
%\left \| u_I-u_h\right \|_{\infty,\Omega_h}\le\frac{1}{8}\left \|\Delta_h u_I-(\Delta u)_I\right \|_{\infty,\Omega_h\setminus \Gamma_h}
%$$
%下一步是研究一致性误差 $\left \|\Delta_h u_I-(\Delta u)_I\right \|_{h,\infty}$ .下面的引理可以很容易地用 $Taylor$ 展式来证明。

%\textbf{引理3.4}
%如果 $u\in C^4(\Omega)$,那么
%$$
%\left \|\Delta_h u_I-(\Delta u)_I\right \|_{\infty,\Omega_h \setminus \Gamma_h}\le\frac{h^2}{6}max\left \{\left \|\frac{\partial ^4u}{\partial x^4}\right \|_{\infty,\Omega},\left \|\frac{\partial ^4u}{\partial y^4}\right \|_{\infty}\right \}
%$$
%在以下定理中,我们总结了有限差分方法的收敛性结果。

%\textbf{定理3.5}
%设 $u$ 是 $Dirichlet$ 问题($4$)的解,以及离散问题($12$)的解。如果 $u\in C^4(\Omega)$,那么
%$$
%\left \| u_I-u_h\right \|_{\infty,\Omega_h}\le Ch^2
%$$
%常数 $C$
%$$
%C=\frac{1}{48}max\left \{\left \|\frac{\partial ^4u}{\partial x^4}\right \|_{\infty,\Omega},\left \|\frac{\partial ^4u}{\partial y^4}\right \|_{\infty}\right \}
%$$
%在实际应用中,即使解 $u$ 比 $C^4(\Omega)$ 不光滑,也有二阶收敛性,即要求 $u\in C^4(\Omega)$ ,这一限制来自于逐点估计。在有限元法中,我们将用积分范数来得到正确的函数空间。





































%坎坎坷坷扩


%\cite{tam19912d}
%\bibliography{../ref}
\end{document}
