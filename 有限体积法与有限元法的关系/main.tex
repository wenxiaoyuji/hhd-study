% !Mode:: "TeX:UTF-8"
\documentclass[12pt,a4paper]{article}
\input{../en_preamble.tex}
\input{../xecjk_preamble.tex}

\title{有限差分法}
%\author{}
\date{\chntoday}

\begin{document}

\section{与有限元法的区别}
我们将回顾有限元方法,并展示有限元方法与有限体积法之间的关系。设 $a(u,v)$ 为双线性形式
$$
a(u,v)=\int_{\Omega}^{} (\textbf{k}\nabla u)\cdot\nabla v dx
$$
线性有限元法是:找到 $u_{L}\in V_T$ 使得
$$
a(u_L,v)=(f,v) ~ for ~ all ~ v\in V_T
$$
对于有限元,试探函数空间和检验函数空间是相同的,就是所谓的 $Galerkin$ 方法。

为了显示它们的关系,我们现在生成相应的矩阵方程。设 $N(T)$ 是 $T$ 的内节点集,$N=N(T)$,然后 $dim ~ V_B=dim ~ V_T=N$.可以选择 $V_B$ 的基作为每个 $b_i,i=1,\cdots ,N$ 的特征函数

$$
\psi _i=\chi _{b_i}(x)=
\begin{cases}
1, & x\in b_i \\
0, & otherwise
\end{cases}
$$

线性有限元空间 $V_T$ 的节点基是标准的帽函数:
$$
\phi _i\in V_T,\phi _i(x_j)=\delta_{ij},\forall x_j\in N(T),~i=1,\cdots ,N
$$
令 $\bar{u}=\sum_{j=1}^N \bar{U}_j\phi_j$,选择 $v=\psi_i,i=1,\cdots ,N$,我们得到一个线性代数方程
$$
\bar{A}\bar{U}=\bar{F}
$$

$$
\bar{A}_{ij}=-\int_{\partial b_i}^{}(\textbf{k}\nabla \phi_j)\cdot\textbf{n},~\bar{F}_i=\int_{b_i}^{} fdx
$$
令 $u_L=\sum_{j=1}^N U_j\phi_j$,选择 $v=\phi_i,i=1,\cdots ,N$,我们得到另外一个线性代数方程
$$
AU=F
$$

$$
A_{ij}=\int_{\Omega}^{}(\textbf{k}\nabla \phi_j)\cdot\nabla \phi_i,~F_i=\int_{\Omega}^{} f\phi_i dx
$$
我们将证明当 $\textbf{K}(x)$ 在每个三角形上是分段常数时,有 $A=\bar{A}$.解向量是 $u_L$ 和 $\bar{u}$ 在顶点处的点值。唯一的区别是计算右手边的方法不同。对于 $FEM$,$F_i=\int_{\omega_i}^{}f\phi_i dx$是顶点的加权平均值。对于 $FVM$,$\bar{F}_i=\int_{b_i}^{}fdx$是控制体积 $b_i$ 的平均值。当我们选择 $A$ 型控制体积时,即选择 $c_{\tau}$ 作为 $\tau$ 的重心,$\bar{F}_i$ 可以被认为是 $F_i$ 的近似值。

\textbf{引理 5.1}
给定 $R^n$ 中具有 $L$ 面的多面体$\Omega$,令 $|F_i|$ 表示面 $F_i$ 的测度,并且 $n_i$ 表示第 $i$ 侧的单位外法线,$i=1,\cdots ,L$.那么
$$
\sum_{i=1}^L |F_i| {n}_i=0
$$
证明:设 $\left\{e_k\right\}$ 是 $R^n$ 的正交基。然后通过散度定理
$$
\sum_{i=1}^L |F_i| e_k\cdot {n}_i=\int_{\partial\Omega}^{} e_k\cdot\textbf{n}ds=\int_{\Omega}^{}div ~ e_k dx=0
$$
下面的定理是非常重要的,即 $FVM$ 和 $FEM$具有刚度矩阵。让我们引入线性空间 $G$ 的同构:通过 $\psi _i\to\phi _i,1\le i\le N$, $V_B\to V_{\tau}$,那么 $u=\sum_{i=1}^N u_i\psi _i\in V_B,G_u=\sum_{i=1}^N u_i\phi _i\in V_{\tau}$.注意,$u$ 和 $G_u$ 有相同的向量 $U=(u_1,\cdots,u_N)^T$.我们还使用简单的符号 $\bar{u}$ 来表示 $G_u$.

\textbf{定理 5.2}
假设 $\textbf{K}(x)$ 在每个 $\tau\in T$上是分段常数,$\partial b_i \cap\partial\tau$ 由边的中间点组成,那么
$$
a(u,v)=\bar{a}(u,\bar{v}),~\forall u,v\in V_{\tau}
$$ 
证明:由于我们假设 $K(x)$是分段常数,我们只需要给出一个三角形上泊松方程的局部刚度。这减少了证明
$$
\bar{a}(\lambda_i,G\lambda_j)=a(\lambda_i,\lambda_j)
$$
具体来说,我们以 $\lambda_1$,$\lambda_2$ 为例。

由于 $\nabla\lambda _1$ 在三角形 $\tau$ 上是一个常数,所以我们可以对它积分,用引理两次得到
$$
-\int_{e_1\cup e_2}^{} \nabla\lambda _1\cdot\textbf{n}ds=-\nabla\lambda _1\cdot (|e_1|\textbf{n}_{e_1}+|e_2|\textbf{n}_{e_2})=\nabla\lambda _1\cdot \frac{1}{2}(|l_1|\textbf{n}_1+|l_3|\textbf{n}_3)=-\nabla\lambda _1\cdot \frac{1}{2}|l_2|\textbf{n}_2=\int_{\tau}^{} \nabla\lambda _1\cdot\nabla\lambda _2
$$
在最后一步中,我们使用了公式 $\nabla\lambda _i$

\section{误差分析}

以顶点为中心的线性有限体积法的误差分析依赖于线性有限元法和线性有限体积法之间的密切关系。与前一节相比,线性 $FVM$ 近似 $\bar{u}$ 可以看作是有限元近似 $u_L$ 的一个扰动。利用这种关系可以得到能量范数的一阶最优收敛速度。

注意,对于 $B$ 型双网格,右侧可能完全不同。例如,设 $f=1$ 并考虑图中的控制体积,那么当 $\bar{F}_i=|\omega_i|/4$ 时,$F_i=|\omega_i|/3$.通过比较 $H^{−1}$ 范数中的最优一阶收敛性,仍然可以得到 $H^1$ 范数的最优一阶收敛性。

此外,如果我们选择 $A$ 型对偶网格,则还可以得到 $L^2$ 范数中的最优二阶收敛性。注意,对于控制体积的一般选择,有限体积近似可能不会导致最优 $L^2$ 范数收敛速度。有关 $B$ 型对偶网格的示例,请参见[?]。通过将其视为有限元方法的扰动,也可以得到最优 $L^\infty$ 范数估计;见[?,?]。我们不打算讨论 $L^2$ 或 $L^\infty$ 误差估计。

\textbf{定理 6.1}
假设 $\textbf K(x)$ 在每个 $\tau\in T$ 上是分段常数,则扩散方程的解 $u$ 在 $H^1_0(\Omega)\cap H^2(\Omega)$ 空间中。并且网格是均匀网格,网格大小为 $h$,则有限体积近似 $u_h$ 具有最佳逼近阶
$$
\left \|u-u_h\right \|_{1,\Omega}\lesssim h(\left \|u\right \|_{2,\Omega}+\left \|f\right \|)
$$
证明:$\forall f\in L^2(\Omega)$,我们将 $\Pi_h f\in V'_{\tau}$ 定义为 
$$
<\Pi_h f,v_h> = (f,Gv_h) ,~\forall v_h\in V_{\tau}
$$
将 $Q_h f\in V'_{\tau}$ 定义为 
$$
<Q_h f,v_h> = (f,v_h) ,~\forall v_h\in V_{\tau}
$$
按照 $Hackbusch$ 的表示法,我们用 $u^G_h$ 表示为标准的 $Galerkin$(有限元)逼近,$u^B_h$ 是 $box$(有限体积)近似。刚度矩阵的等价意味着
$$
L_h u^G_h=Q_hf,L_h u^B_h=\Pi_h f
$$		
因此,由于 $L^{-1}_h$ 的稳定性,我们有
$$
\left| \ u^G_h-u^B_h \right|_1=\sup_{v_h\in V_{\tau}}\frac{<Q_h f-\Pi_h f,v_h>}{|v_h|_1}
$$

根据定义
$$
<Q_h f-\Pi_h f,v_h>=(f,v_h-Gv_h)
$$
表示位于 $x_i$ 的 $hat$ 基函数为 $\omega _i$.注意,$b_i\subset ω_i$ 和运算符 $I-G$ 在 $\omega _i$ 中保留常数函数,因此 
$$
(f,v_h-Gv_h)_{b_i}\le \left \| \ f\right \|_{b_i}\left \| \ v_h-Gv_h\right \|_{\omega _i}\le Ch^2\left \| \ f \right \|_{b_i}\left|\ v_h \right|_{1,\omega _i}
$$
综上所述,并利用柯西·施瓦兹不等式,我们得到了一阶收敛性
$$
\left|\ u^G_h-u^B_h \right|_1\le Ch\left \|\ f \right \|
$$
该估计是由三角形不等式和有限元法的估计得到的。










%\input{test.tex}

%\cite{tam19912d}
%\bibliography{../ref}
\end{document}
