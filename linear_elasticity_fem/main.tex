% !Mode:: "TeX:UTF-8"
\documentclass[12pt,a4paper]{article}
\input{../en_preamble.tex}
\input{../xecjk_preamble.tex}

\title{线弹性有限元解法}
%\author{}
\date{\chntoday}

\begin{document}
\maketitle

\section{变分}

给定一个向量测试函数空间 $V$,
$$
-\int_{\Omega}^{} (\nabla\cdot\boldsymbol{\sigma})\cdot \boldsymbol{v}\, \mathrm{d}x=\int_{\Omega}^{} \boldsymbol{f}\cdot \boldsymbol{v}\, \mathrm{d}x, ~~ v\in V
$$

因为 $\nabla \cdot\boldsymbol{\sigma}=\begin{bmatrix}
\frac{\partial\sigma_x}{\partial x}+\frac{\partial\tau_{yx}}{\partial y}+\frac{\partial\tau_{zx}}{\partial z} \\
\frac{\partial\tau_{xy}}{\partial x}+\frac{\partial\sigma_{y}}{\partial y}+\frac{\partial\tau_{zy}}{\partial z} \\
\frac{\partial\tau_{xz}}{\partial x}+\frac{\partial\tau_{yz}}{\partial y}+\frac{\partial\sigma_{x}}{\partial z}
\end{bmatrix}, ~ \boldsymbol{v}=
\begin{bmatrix}
v_x \\
v_y \\
v_z
\end{bmatrix}, ~ \boldsymbol{f}=
\begin{bmatrix}
f_x \\
f_y \\
f_z
\end{bmatrix}$,所以有
$$
-\int_{\Omega}^{} \begin{bmatrix}
\frac{\partial\sigma_x}{\partial x}+\frac{\partial\tau_{yx}}{\partial y}+\frac{\partial\tau_{zx}}{\partial z} \\
\frac{\partial\tau_{xy}}{\partial x}+\frac{\partial\sigma_{y}}{\partial y}+\frac{\partial\tau_{zy}}{\partial z} \\
\frac{\partial\tau_{xz}}{\partial x}+\frac{\partial\tau_{yz}}{\partial y}+\frac{\partial\sigma_{x}}{\partial z}
\end{bmatrix}\cdot\begin{bmatrix}
v_x \\
v_y \\
v_z
\end{bmatrix}\, \mathrm{d}x=
\int_{\Omega}^{} \begin{bmatrix}
f_x \\
f_y \\
f_z
\end{bmatrix}\cdot \begin{bmatrix}
v_x \\
v_y \\
v_z
\end{bmatrix}\, \mathrm{d}x, ~~ v\in V
$$
因此
$$
-\int_{\Omega}^{} (\frac{\partial\sigma_x}{\partial x}+\frac{\partial\tau_{yx}}{\partial y}+\frac{\partial\tau_{zx}}{\partial z})v_x + (\frac{\partial\tau_{xy}}{\partial x}+\frac{\partial\sigma_{y}}{\partial y}+\frac{\partial\tau_{zy}}{\partial z})v_y +(\frac{\partial\tau_{xz}}{\partial x}+\frac{\partial\tau_{yz}}{\partial y}+\frac{\partial\sigma_{x}}{\partial z})v_z \, \mathrm{d}x=
\int_{\Omega}^{} f_x v_x + f_y v_y +f_z v_z \, \mathrm{d}x
$$
分部积分可得
$$
-\int_{\Omega}^{}(\nabla\cdot\boldsymbol{\sigma})\cdot\boldsymbol{v}\, \mathrm{d}x=
\int_{\Omega}^{}\boldsymbol{\sigma}:\nabla\boldsymbol{v} \, \mathrm{d}x-\int_{\partial\Omega}^{} (\boldsymbol{\sigma}\cdot\boldsymbol{n})\cdot\boldsymbol{v} \, \mathrm{d}x
$$
从而
$$
\int_{\Omega}^{}\boldsymbol{\sigma}:\nabla v \, \mathrm{d}x=\int_{\Omega}^{}\boldsymbol{f}\cdot\boldsymbol{v}\, \mathrm{d}x+\int_{\partial\Omega _g}^{}\boldsymbol{g}\cdot\boldsymbol{v}\, \mathrm{d}x
$$
其中 $\boldsymbol{g}=\boldsymbol{\sigma}\cdot\boldsymbol{n}$ 为边界 $\partial\Omega _g$ 的边界条件

$:$ 是两个张量之间的内积,对二维张量来说,就是对应元素相乘再相加。

因为一个对称张量和一个反对称张量的内积为 $0$,
$$
(\nabla\boldsymbol{v}-\nabla\boldsymbol{v}^T)^T=\nabla\boldsymbol{v}^T-\nabla\boldsymbol{v}=-(\nabla\boldsymbol{v}-\nabla\boldsymbol{v}^T)
$$
即 $\nabla\boldsymbol{v}-\nabla\boldsymbol{v}^T$ 是反对称矩阵,因此
$$
\int_{\Omega}^{} \boldsymbol{\sigma}:(\nabla\boldsymbol{v}-\nabla\boldsymbol{v}^T) \, \mathrm{d}x = 0
$$
$$
\int_{\Omega}^{} \boldsymbol{\sigma}:\nabla\boldsymbol{v} \, \mathrm{d}x =\int_{\Omega}^{} \boldsymbol{\sigma}:\nabla\boldsymbol{v}^T \, \mathrm{d}x
$$
$$
\int_{\Omega}^{}\boldsymbol{\sigma}(\boldsymbol{u}):\frac{\nabla\boldsymbol{v}+\nabla\boldsymbol{v}^T}{2} \, \mathrm{d}x=\int_{\Omega}^{}\boldsymbol{f}\cdot\boldsymbol{v}\, \mathrm{d}x+\int_{\partial\Omega _g}^{}\boldsymbol{g}\cdot\boldsymbol{v}\, \mathrm{d}x
$$
所以上面的变分形式还可以变为
$$
\int_{\Omega}^{}\boldsymbol{\sigma}(\boldsymbol{u}):\boldsymbol{\varepsilon}(\boldsymbol{v}) \, \mathrm{d}x=\int_{\Omega}^{}\boldsymbol{f}\cdot\boldsymbol{v}\, \mathrm{d}x+\int_{\partial\Omega _g}^{}\boldsymbol{g}\cdot\boldsymbol{v}\, \mathrm{d}x
$$

\section{基函数}

现在我们要把位移的每一个分量用线性函数来表示,其中 $\varphi_1,\varphi_2,\varphi_3,\varphi_4$ 是线性函数。
$$
u=u_1\varphi_1+u_2\varphi_2+u_3\varphi_3+u_4\varphi_4
$$
$$
v=v_1\varphi_1+v_2\varphi_2+v_3\varphi_3+v_4\varphi_4
$$
$$
w=w_1\varphi_1+w_2\varphi_2+w_3\varphi_3+w_4\varphi_4
$$
即
$$
\begin{bmatrix}
u \\
v \\
w
\end{bmatrix}=\varphi_1\begin{bmatrix}
u_1 \\
v_1 \\
w_1
\end{bmatrix}+\varphi_2\begin{bmatrix}
u_2 \\
v_2 \\
w_2
\end{bmatrix}+\varphi_3\begin{bmatrix}
u_3 \\
v_3 \\
w_3
\end{bmatrix}+\varphi_4\begin{bmatrix}
u_4 \\
v_4 \\
w_4
\end{bmatrix}
$$
$\varphi_1,\varphi_2,\varphi_3,\varphi_4$ 又称为关于点 $\boldsymbol{u}$ 的重心坐标。











































































%\cite{tam19912d}
%\bibliography{../ref}
\end{document}
